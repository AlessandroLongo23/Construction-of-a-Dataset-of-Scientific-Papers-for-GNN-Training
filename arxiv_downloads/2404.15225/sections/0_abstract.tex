\begin{abstract}\label{abstract}
Link prediction (LP), inferring the connectivity between nodes, is a significant research area in graph data, where a link represents essential information on relationships between nodes. 
Although graph neural network (GNN)-based models have achieved high performance in LP, understanding why they perform well is challenging because most comprise complex neural networks. 
We employ persistent homology (PH), a topological data analysis method that helps analyze the topological information of graphs, to explain the reasons for the high performance. 
We propose a novel method that employs PH for LP (PHLP) focusing on how the presence or absence of target links influences the overall topology. 
The PHLP utilizes the \textit{angle hop subgraph} and new node labeling called \textit{degree double radius node labeling (Degree DRNL)}, distinguishing the information of graphs better than DRNL. Using only a classifier, PHLP performs similarly to state-of-the-art (SOTA) models on most benchmark datasets. 
Incorporating the outputs calculated using PHLP into the existing GNN-based SOTA models improves performance across all benchmark datasets. 
To the best of our knowledge, PHLP is the first method of applying PH to LP without GNNs. 
The proposed approach, employing PH while not relying on neural networks, enables the identification of crucial factors for improving performance.
\end{abstract}
