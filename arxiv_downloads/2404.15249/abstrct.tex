\begin{abstract}
   % The kernel-free boundary integral (KFBI) method has successfully solved partial differential equations (PDEs) on irregular domains. All present KFBI methods are implemented on the CPU platform, and this paper presents the algorithms of single-GPU and multiple-GPUs. The KFBI is a Cartesian grid method, which is inherently suitable for GPU parallel computing. On a single GPU, assigning individual threads can control correction, interpolation, and jump calculations. To facilitate the computation of larger-scale problems,  we extend the algorithm to multiple GPUs. Unlike a single GPU, we use the arrowhead decomposition method to solve the interface problem, achieving optimal computing efficiency and load balancing. Numerical examples show that the proposed algorithm is second-order accurate and efficient. Single-GPU solver speeds 50-200 times than traditional CPU while the eight GPUs distributed solver yields up to $60\%$ parallel efficiency.

  The kernel-free boundary integral (KFBI) method has successfully solved partial differential equations (PDEs) on irregular domains. Diverging from traditional boundary integral methods, the computation of boundary integrals in KFBI is executed through the resolution of equivalent simple interface problems on Cartesian grids, utilizing fast algorithms. While existing implementations of KFBI methods predominantly utilize CPU platforms, GPU architecture's superior computational capabilities and extensive memory bandwidth offer an efficient resolution to computational bottlenecks. This paper delineates the algorithms adapted for both single-GPU and multiple-GPU applications. On a single GPU, assigning individual threads can control correction, interpolation, and jump calculations. The algorithm is expanded to multiple GPUs to enhance the processing of larger-scale problems. The arrowhead decomposition method is employed in multiple-GPU settings, ensuring optimal computational efficiency and load balancing. Numerical examples show that the proposed algorithm is second-order accurate and efficient. Single-GPU solver speeds 50-200 times than traditional CPU while the eight GPUs distributed solver yields up to $60\%$ parallel efficiency.
      % The kernel-free boundary integral (KFBI) method has successfully solved partial differential equations (PDEs) on irregular domains. Diverging from conventional boundary integral methods, the computation of boundary integrals in KFBI is executed through the resolution of equivalent simplified interface problems on Cartesian grids, utilizing fast algorithms.
      % %Hence, it does not need to know any analytical expression of the fundamental solution or Green’s function in evaluation of boundary or volume integrals. 
      % % All present KFBI methods are implemented on the CPU platform. Due to the GPU architecture’s excellent computing ability and large memory bandwidth, it becomes an efficient solution to solve computing bottlenecks. 
      % While existing implementations of KFBI methods predominantly utilize CPU platforms, the superior computational capabilities and extensive memory bandwidth of GPU architecture offer an efficient resolution to computational bottlenecks. This paper delineates the algorithms adapted for both single-GPU and multiple-GPUs applications.  On a single GPU, assigning individual threads can control correction, interpolation, and jump calculations.The algorithm is expanded to multiple GPUs to enhance the processing of larger-scale problems. the arrowhead decomposition method is employed in multi-GPU settings, ensuring optimal computational efficiency and load balancing. Numerical examples show that the proposed algorithm is second-order accurate and efficient. Single-GPU solver speeds 50-200 times than traditional CPU while the eight GPUs distributed solver yields up to $60\%$ parallel efficiency.

      
% The Kernel-Free Boundary Integral (KFBI) method has been effectively utilized for solving partial differential equations (PDEs) in irregular domains. Diverging from conventional boundary integral methods, the computation of boundary integrals in KFBI is executed through the resolution of equivalent simplified interface problems on Cartesian grids, utilizing rapid algorithms. This approach eliminates the need for analytical expressions of fundamental solutions or Green's functions in the evaluation of boundary or volume integrals.

% While existing implementations of KFBI methods predominantly utilize CPU platforms, the superior computational capabilities and extensive memory bandwidth of GPU architecture offer an efficient resolution to computational bottlenecks. This paper delineates the algorithms adapted for both single-GPU and multiple-GPUs applications. On a single GPU, individual threads are designated for managing correction, interpolation, and jump calculations. The algorithm is expanded to multiple GPUs to enhance the processing of larger-scale problems. Contrary to the single GPU approach, the arrowhead decomposition method is employed in multi-GPU settings, ensuring optimal computational efficiency and load balancing. Numerical examples demonstrate the proposed algorithm's second-order accuracy and efficiency, with the single-GPU solver achieving speeds 50-200 times faster than traditional CPU implementations. Additionally, the eight-GPU distributed solver attains up to 60% parallel efficiency.
\end{abstract}