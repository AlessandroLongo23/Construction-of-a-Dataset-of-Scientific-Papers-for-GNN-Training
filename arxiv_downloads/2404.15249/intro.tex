\section{Introduction}



Graphics Processing Units (GPU) are co-processors originally devoted to accelerate graphics processing. In the last years, they are extensively used as massively parallel platforms to run general-purpose programs. This practice is mostly known as General-Purpose computing on Graphics Processing Units (GPGPU). This growing trend is confirmed by the number of computers in the top500 ranking that are provided of GPUs, which on November 2023 was 186\cite{Top500}.

One of the areas taking advantage of the capabilities of this kind of accelerators is scientific computing. There are many recent publications describing works that successfully port code from CPU to GPU, achieving important speedups\cite{Navarro2014,Hwu2012}. Elliptic type problems are widely applied in the fields of electrochemistry\cite{QIAN2021109908,DING2019108864}, electromagnetism\cite{Chai2023}, computational fluid dynamics\cite{Greengard1998318, Quartapelle1993NumericalSO}, shape optimisation problems\cite{ZHU2011752,gong2023} and other areas in science\cite{Chapko199747,ZHOU20061,CHENG2006616,SUN2014445},  Solving these problems often requires a substantial computational cost\cite{GarciaRisueno2014}.

An effective and accurate approach for solving elliptical equations is the Kernel-Free Boundary Integral (KFBI) method \cite{ying2007kernel, ying2013kernel, ying2014kernel}, which originates from boundary integral methods. Unlike traditional boundary integral approaches, the KFBI method embeds complex domains into larger, regular computational areas (such as square regions), which are subsequently partitioned using Cartesian grids. The KFBI method not only benefits from the well-conditioning property of the boundary integral equation(BIE) but also avoids explicitly calculating Green's function directly, which is challenging in complex domains\cite{xie2019fourth,ying2013kernel}.
In recent years, the KFBI method has been extensively applied\cite{xie2019fourth,xie2021cartesian,ZHAO2023116163,dong2023kernelfree,zhou2023adi}.
% An efficient and accurate way to solve Elliptic equation is
% the KFBI method\cite{ying2007kernel,ying2013kernel,ying2014kernel}. As an extension of Mayo's boundary integral method\cite{mayo1984fast,mayo1985fast,mayo1992rapid}, the KFBI is a Cartesian grid approach to solve general Elliptic PDEs in smooth irregular domains. Unlike other Cartesian methods such as the decomposed II method\cite{berthelsen2004decomposed} and the matched interface and boundary method\cite{zhou2006high}, the KFBI method distinguishes itself by preserving the symmetric and positive definiteness of the resulting discrete system. This unique feature enables the application of efficient solution techniques, such as FFT-based solvers or geometric multigrid solvers. Derived from the boundary integral method, the KFBI method not only benefits from the well-conditioning property of the boundary integral equation(BIE) but also avoids explicitly calculating Green's function directly, which is challenging in complex domains\cite{xie2019fourth,ying2013kernel}.
% In recent years, the KFBI method has been extensively applied\cite{xie2019fourth,xie2021cartesian,ZHAO2023116163,dong2023kernelfree,zhou2023adi}.

% Unlike the traditional boundary integral method, the KFBI method embeds the complex domain into a larger regular computational region. It reformulates to solve equivalent interface problem instead of directly evaluating the volume and boundary integrals \cite{ying2007kernel}. The boundary integral method, in the literature, leads to a dense $N \times N$ discrete linear system(N represents the degrees of freedom(DOF) defined on the boundary), which successfully reduces the dimensionality of the model and precisely captures the complex boundary\cite{xie2019fourth}. Although the DOF of the KFBI method is also O(N), the resulting $N \times N$ linear system of the interface problem is sparse, symmetric, and positive definite. Despite using a fast solver, the single CPU algorithm of the KFBI method is still time-consuming and dissatisfies larger-scale calculations. The primary purpose of this paper is to propose an efficient, GPU-based algorithm for the KFBI method, including implementation details on single GPU and multiple GPUs.
% Elliptic problems are widely applied in the fields of electrochemistry \cite{QIAN2021109908,DING2019108864}, electromagnetism\cite{Chai2023}, computational fluid dynamics\cite{Greengard1998318, Quartapelle1993NumericalSO}, shape optimisation problems\cite{ZHU2011752,gong2023} and other areas in science \cite{Chapko199747,ZHOU20061,CHENG2006616,SUN2014445}. Representative methods for solving specific elliptic problems are the finite difference method\cite{zhou2006high, fedkiw1999non, leveque1994immersed, mccorquodale2001cartesian, berthelsen2004decomposed,johansen1998cartesian,twizell1996second,wiegmann2000explicit,hou2012numerical}, finite element method \cite{li2003newinterface,li1999fem,HUANG2017439,chu2010, wen2018finite, HOU2005411} , boundary integral method\cite{jwason1963,YING2004591,YING2006247,greengard_rokhlin_1997}, and the deep learning method\cite{E2018deepRitz,RAISSI2019686,lulu2021deeponet,li2021fourier,Zang_2020,fan2023decoupling,HU2022111576,HE2022114358,lin2021binet,lin2022bigreennet}, which becomes very popular in recent years. As a competitive approach to traditional  and new methods, the KFBI method proposed by Ying\cite{ying2007kernel,ying2014kernel,ying2013kernel} has shown its advantages in recent years.

% As an extension of Mayo's boundary integral method\cite{mayo1984fast,mayo1985fast,mayo1992rapid}, the KFBI is a Cartesian grid approach to solve general elliptic PDEs in smooth irregular domains. Unlike other Cartesian methods such as the decomposed II method\cite{berthelsen2004decomposed} and the matched interface and boundary method\cite{zhou2006high}, the KFBI method distinguishes itself by preserving the symmetric and positive definiteness of the resulting discrete system. This unique feature enables the application of efficient solution techniques, such as FFT-based solvers or geometric multigrid solvers. Derived from the boundary integral method, the KFBI method not only benefits from the well-conditioning property of the boundary integral equation(BIE) but also avoids explicitly calculating Green's function directly, which is challenging in complex domains\cite{xie2019fourth,ying2013kernel}.

% The KFBI method has shown promise in recent years. Xie further developed the fourth order and its compact schemes\cite{xie2021cartesian,xie2019fourth}. Dong and Zhao successfully applied the KFBI method to solve Stokes equation \cite{ZHAO2023116163}, two-phase Stokes with  discontinuity\cite{dong2023kernelfree} and prove second order convergence of MAC scheme\cite{dong2023kernelfree,dong2023second}. Zhou introduced a novel second-order KFBI-ADI method, employing the 1D KFBI method to solve problems in multiple spatial dimensions\cite{zhou2023adi}. Yang extends the KFBI method to unbounded interface problems with source term\cite{jhyang2023}. 
%Additionally, Tan expanded the KFBI method to the discrete heat, wave, and Schrödinger equations through the time-stepping method[]. 

% The KFBI method has shown promise in recent years when compared to other Cartesian grid methods like the ghost fluid method\cite{fedkiw1999non}, the immersed boundary method\cite{leveque1994immersed}, and the volume of fluid method\cite{mccorquodale2001cartesian}. Xie further developed fourth order and its compact schemes\cite{xie2021cartesian,xie2019fourth}. Dong and Zhao successfully applied the KFBI method to solve Stokes equation \cite{ZHAO2023116163}, two-phase Stokes with  discontinuity\cite{dong2023kernelfree} and prove second order convergence of MAC scheme\cite{dong2023kernelfree,dong2023second}. Zhou introduced a novel second-order KFBI-ADI method, employing the 1D KFBI method to solve problems in multiple spatial dimensions\cite{zhou2023adi}. Yang extend the KFBI method to unbounded interface problem with source term\cite{jhyang2023}. Additionally, Tan expanded the KFBI method to the discrete heat, wave, and Schrödinger equations through the time-stepping method[]. 

% The motivation of this paper is to provide an efficient, scalable, GPU-based KFBI algorithm as an alternative to the existing finite difference method(FDM), the finite element method(FEM), the boundary element method(BEM), and the neural network method(NNM) for solving such partial differential equations.

% The finite difference method\cite{zhou2006high, fedkiw1999non, leveque1994immersed, mccorquodale2001cartesian, berthelsen2004decomposed,johansen1998cartesian,twizell1996second,wiegmann2000explicit,hou2012numerical} is suitable for structure grid and result linear system can be solved by fast solvers such as FFT-based solver or multigrid solver, but for grid lines that are not aligned with the boundaries or interface of discontinuities, which generally degrades the accuracy\cite{ying2007kernel}.Though finite element method \cite{li2003newinterface,li1999fem,HUANG2017439,chu2010, wen2018finite, HOU2005411} method is flexibly applied for irregular regions, and the resulting linear system may not always be symmetrical. The numerical solution is closely related to the mesh generation, which sometimes is time-consuming and challenging\cite{wen2018finite,ying2007kernel}. The neural network method \cite{E2018deepRitz,RAISSI2019686,lulu2021deeponet,li2021fourier,Zang_2020,fan2023decoupling,HU2022111576,HE2022114358,lin2021binet,lin2022bigreennet} is a meshfree method that is convenient to compute in complex or irregular domains\cite{li2021fourier,Zang_2020,HE2022114358}, degradation phenomenon\cite{fan2023decoupling}, and discontinuity on boundary\cite{fan2023decoupling,HU2022111576,HE2022114358}. However, neural network methods are challenging to control the accuracy of numerical algorithms and need more comprehensive theoretical analysis.\cite{fan2023decoupling}. The BIM\cite{jwason1963,YING2004591,YING2006247,greengard_rokhlin_1997} method reduced dimensionality of the model and the ease of complex boundary capturing are two evident merits. However, it requires the display of the expression Green's function and is difficult to handle nonlinear problems in a straightforward manner\cite{xie2019fourth}.

% Inspired by Mayo's work, Ying has proposed the KFBI method\cite{ying2014kernel,ying2013kernel,ying2007kernel,xie2021cartesian,xie2019fourth, ZHAO2023116163,dong2023kernelfree,dong2023second} for elliptic boundary value problems as a competitive alternative to other Cartesian grid methods such as the high-order matched interface and boundary method\cite{zhou2006high},  the ghost fluid method\cite{fedkiw1999non}, the immersed boundary(IB) method\cite{leveque1994immersed}, the volume of fluid(VOF) method \cite{mccorquodale2001cartesian}, the decomposed II method \cite{berthelsen2004decomposed} and so on\cite{johansen1998cartesian,twizell1996second,wiegmann2000explicit,hou2012numerical}. Xie developed fourth order and its compact schemes\cite{xie2021cartesian,xie2019fourth}. Dong and Zhao applied the KFBI method for solving Stokes equation \cite{ZHAO2023116163}, two-phase Stokes with  discontinuity\cite{dong2023kernelfree} and prove second order convergence of MAC scheme\cite{dong2023kernelfree,dong2023second}. Compared to II, IB, and Mayo's methods, the KFBI method can handle more general elliptic operators with possible anisotropy and inhomogeneity\cite{ying2007kernel}. Unlike the GF and MIB methods, the KFBI method can keep the solution matrix after discretization symmetrically positive definite, thus allowing it to be solved using fast algorithms\cite{ying2007kernel}. 

% Different from FDM or FEM, the KFBI method embeds the complex domain into a larger regular domain and reformulate to solve the equivalent interface problem iteratively. In the literature, BIM can reduce the dimensionality of the model and exactly capture the complex boundary\cite{xie2019fourth}, which leads to a dense N \times N discreting linear system(N represents the degree of freedoms defined on the boundary). However, compared with traditional BIM, the KFBI method introduces the interface problem in every iteration, which increases more degrees of freedoms(DOFs). In fact, the cost of each interface problem depends on the ways of partitioning into hierarchy of structure grid. The main purpose of this paper is to propose a efficient, GPU-based algorithm for the KFBI method, including single GPU and multi-GPU. 

% Unlike the boundary integral method, the KFBI method embeds the complex domain into a larger regular computational region. It reformulates to solve the equivalent interface problem instead of directly evaluating the volume and boundary integral \cite{ying2007kernel}. The boundary integral method, in the literature, leads to a dense $N \times N$ discrete linear system(N represents the degrees of freedom(DOF) defined on the boundary), which successfully reduces the dimensionality of the model and precisely captures the complex boundary\cite{xie2019fourth}. Although the DOF of the KFBI method is also O(N), the resulting $N \times N$ linear system of the interface problem is sparse, symmetric, and positive definite. Despite using a fast solver, the single CPU algorithm of the KFBI method is still time-consuming and dissatisfies larger-scale calculations. The primary purpose of this paper is to propose an efficient, GPU-based algorithm for the KFBI method, including implementation details on single GPU and multiple GPUs. 

% is solved by fast solver such as FFT, it adds more computational cost than the BIM. on the embedded Cartesian grid over BEM and the fixed point iteration process over other FDM methods.

% In order to utilize the KFBI  more efficiently, the focus of this paper is to implement the KFBI method on the GPU platform, including single GPU and multi-GPU\cite{ZHAO2023116163,ying2013kernel}.

\iffalse
While the GPU has a lot of potential as a parallel device, it was
close to useless for scientists for a long time. Apart from accelerated
visualizations, a GPU was not suitable for general-purpose computing. Some early attempts were made for the solution of dense linear
systems (Galoppo et al., 2005), but the development efforts were
huge, since all mathematical computations had to be translated into
graphics operations. However, in 2006 NVIDIA provided a toolkit
which extends the C programming language with a minimal set of primitives for parallelism which allows to use GPUs for general-purpose computing. This extension is called the Compute Unified Device Architecture or CUDA.
\fi
%Recently, there has been a rapid increase in the popularity of GPU (graphics processing unit) computing, particularly following the introduction of new programming languages such as compute unified device architecture (CUDA) [1].
%This surge in interest can be attributed primarily to the remarkable computing capability and superior memory bandwidth inherent in GPU architectures. Consequently, GPU computing has been recognized as a cost-efficient solution for mitigating performance bottlenecks. Numerous applications have thus far experienced significant speedup by harnessing the processing power offered by GPUs. 

% Many scholars have developed a profound interest at GPUs in high-performance calculation. The heightened interest can be attributed to GPU architectures' remarkable computational capabilities and exceptional memory bandwidth, rendering them efficient solutions for addressing performance bottlenecks. Concurrently, a noteworthy trend has emerged within the realm of supercomputing, as evidenced by the Top500 supercomputer rankings, where a substantial majority of supercomputers now embrace heterogeneous architectures. This trend provides a fertile ground for leveraging GPUs to tackle large-scale computational challenges. 
%\textcolor{red}{Indeed, the Cartesian grid methods are well-suited for parallel computing on GPUs. Firstly, Cartesian grids have a straightforward advantage in data representation, where each grid node is typically assigned to a dedicated thread for computation, resulting in high parallelism. Secondly, Cartesian grids possess superficial topological relationships, often relying solely on self and neighboring point information during the computation process. This characteristic is conducive to efficient data transfer, communication, and memory sharing.} 

The substantial memory bandwidth and abundant cores in GPU architecture enable the concurrent execution of thousands of computational tasks, leading to significant acceleration. This renders it an efficient solution for addressing computing bottlenecks. More importantly, the GPU architecture suits the Cartesian grid method since each thread easily controls one grid node. Several related works have addressed the GPU acceleration of Cartesian grid methods in the last ten years\cite{REDDY2015287,cuIBM2018Layton, LIANG2014156Solving, huang2015implementation}: the GPU-accelerated VOF by Rajesh Reddy and R. Banerjee\cite{REDDY2015287}, the CUDA-Based IB method by S. K. Layton, A. Krishnan and L. A. Barba\cite{cuIBM2018Layton}, the TVD Runge–Kutta method on multiple GPUs by Liang. S,  Liu. W and Yuan. L\cite{LIANG2014156Solving}, the multiple-GPU based lattice Boltzmann algorithm by Huang. C, Shi. B, He. N and Chai. Z\cite{huang2015implementation}.

As a Cartesian grid method, the essential procedure of the KFBI method involves the correction of irregular points and control points on the interface individually, making it inherently well-suited for GPU-accelerated parallel processing. Furthermore, the KFBI method utilizes an FFT-based solver, well-documented in literature for its suitability with GPU or GPU clusters\cite{Volkov2011,Govindaraju2008,Nandapalan2012,Chen2010,Nukada2012}, to enhance the efficiency of interface problem computations in iterative procedures. In fact, due to the simple grid topology on Cartesian grids, building a highly parallel GPU-accelerated Cartesian grid solver based on the KFBI method is straightforward. The implementation details of the KFBI solver for a single-GPU version are concisely delineated in section $\ref{oneGPU}$, with the corresponding numerical results presented in section $\ref{result}$.


A significant limitation in single-GPU computation is its available memory, which leads to a bottleneck in the size of the computational mesh. In order to expand the calculation scale and improve efficiency, we also study the multiple-GPU architecture in a single node (in one computer), which contains a two-level parallelization: the coarse-grained level composed by GPUs across multiple CPUs at the cost of coordinating GPU-GPU communication via MPI and the fine-grained level formed by CUDA cores on each GPU. Based on the characteristics mentioned above, we have devised a distributed KFBI algorithm that evenly distributes data to each GPU, maximizing the utilization of multiple-GPU parallel capabilities, ensuring computational load balancing, and minimizing inter-GPU communication overhead.



The remainder of the paper is organized as follows. We first introduce  the boundary integral method and the KFBI methods in section $\ref{KFBI}$. Section $\ref{oneGPU}$ describes implementing the KFBI method on a single GPU. The algorithm is then extended to multiple GPUs and summarised in section $\ref{multi_GPU}$. The numerical results are presented in section $\ref{result}$. The advantages, limitations, and prospects for the GPU-based KFBI method are discussed in the final section. 
