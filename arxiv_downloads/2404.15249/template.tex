\documentclass[mathpazo]{cicp}


%%%%% author macros %%%%%%%%%
% place your own macros HERE
%%%%% end %%%%%%%%%

\begin{document}
%%%%% title : short title may not be used but TITLE is required.
% \title{TITLE}
% \title[short title]{TITLE}
\title{A GPU-accelerated Cartesian grid method for PDEs on irregular domain}

%%%%% author(s) :
% single author:
% \author[name in running head]{AUTHOR\corrauth}
% [name in running head] is NOT OPTIONAL, it is a MUST.
% Use \corrauth to indicate the corresponding author.
% Use \email to provide email address of author.
% \footnote and \thanks are not used in the heading section.
% Another acknowlegments/support of grants, state in Acknowledgments section
% \section*{Acknowledgments}
% \author[O.~Author]{Only Author\corrauth}
% \address{School of Mathematical Sciences, Beijing Normal University,
% Beijing 100875, P.R. China}
% \email{{\tt author@email} (O.~Author)}

% multiple authors:
% Note the use of \affil and \affilnum to link names and addresses.
% The author for correspondence is marked by \corrauth.
% use \emails to provide email addresses of authors
% e.g. below example has 3 authors, first author is also the corresponding
%      author, author 1 and 3 having the same address.
\author[Liwei Tan et.~al.]{Liwei Tan\affil{1}\footnotemark[2],
      Minsheng Huang\affil{1}\footnote[2]{These authors contributed equally to this work.}, and Wenjun Ying\affil{2}\comma\corrauth}
\address{\affilnum{1}\ School of Mathematical Sciences, 
            Shanghai Jiao Tong University, 
            Shanghai 200240, P.R. China. \\
          \affilnum{2}\ School of Mathematical Sciences, MOE-LSC and Institute of Natural Sciences, 
          Shanghai Jiao Tong University, Minhang, 
          Shanghai 200240, P.R. China.}
\emails{{\tt wying@sjtu.edu.cn} (W.~Ying)}

% \author[Zhang Z R et.~al.]{Zhengru Zhang\affil{1}\comma\corrauth,
%       Author Chan\affil{2}, and Author Zhao\affil{1}}
% \address{\affilnum{1}\ School of Mathematical Sciences,
%          Beijing Normal University,
%          Beijing 100875, P.R. China. \\
%           \affilnum{2}\ Department of Mathematics,
%           Hong Kong Baptist University, Hong Kong SAR}
% \emails{{\tt zhang@email} (Z.~Zhang), {\tt chan@email} (A.~Chan),
%          {\tt zhao@email} (A.~Zhao)}
% \footnote and \thanks are not used in the heading section.
% Another acknowlegments/support of grants, state in Acknowledgments section
% \section*{Acknowledgments}

%%%%% Begin Abstract %%%%%%%%%%%
\begin{abstract}
   % The kernel-free boundary integral (KFBI) method has successfully solved partial differential equations (PDEs) on irregular domains. All present KFBI methods are implemented on the CPU platform, and this paper presents the algorithms of single-GPU and multiple-GPUs. The KFBI is a Cartesian grid method, which is inherently suitable for GPU parallel computing. On a single GPU, assigning individual threads can control correction, interpolation, and jump calculations. To facilitate the computation of larger-scale problems,  we extend the algorithm to multiple GPUs. Unlike a single GPU, we use the arrowhead decomposition method to solve the interface problem, achieving optimal computing efficiency and load balancing. Numerical examples show that the proposed algorithm is second-order accurate and efficient. Single-GPU solver speeds 50-200 times than traditional CPU while the eight GPUs distributed solver yields up to $60\%$ parallel efficiency.

  The kernel-free boundary integral (KFBI) method has successfully solved partial differential equations (PDEs) on irregular domains. Diverging from traditional boundary integral methods, the computation of boundary integrals in KFBI is executed through the resolution of equivalent simple interface problems on Cartesian grids, utilizing fast algorithms. While existing implementations of KFBI methods predominantly utilize CPU platforms, GPU architecture's superior computational capabilities and extensive memory bandwidth offer an efficient resolution to computational bottlenecks. This paper delineates the algorithms adapted for both single-GPU and multiple-GPU applications. On a single GPU, assigning individual threads can control correction, interpolation, and jump calculations. The algorithm is expanded to multiple GPUs to enhance the processing of larger-scale problems. The arrowhead decomposition method is employed in multiple-GPU settings, ensuring optimal computational efficiency and load balancing. Numerical examples show that the proposed algorithm is second-order accurate and efficient. Single-GPU solver speeds 50-200 times than traditional CPU while the eight GPUs distributed solver yields up to $60\%$ parallel efficiency.
      % The kernel-free boundary integral (KFBI) method has successfully solved partial differential equations (PDEs) on irregular domains. Diverging from conventional boundary integral methods, the computation of boundary integrals in KFBI is executed through the resolution of equivalent simplified interface problems on Cartesian grids, utilizing fast algorithms.
      % %Hence, it does not need to know any analytical expression of the fundamental solution or Green’s function in evaluation of boundary or volume integrals. 
      % % All present KFBI methods are implemented on the CPU platform. Due to the GPU architecture’s excellent computing ability and large memory bandwidth, it becomes an efficient solution to solve computing bottlenecks. 
      % While existing implementations of KFBI methods predominantly utilize CPU platforms, the superior computational capabilities and extensive memory bandwidth of GPU architecture offer an efficient resolution to computational bottlenecks. This paper delineates the algorithms adapted for both single-GPU and multiple-GPUs applications.  On a single GPU, assigning individual threads can control correction, interpolation, and jump calculations.The algorithm is expanded to multiple GPUs to enhance the processing of larger-scale problems. the arrowhead decomposition method is employed in multi-GPU settings, ensuring optimal computational efficiency and load balancing. Numerical examples show that the proposed algorithm is second-order accurate and efficient. Single-GPU solver speeds 50-200 times than traditional CPU while the eight GPUs distributed solver yields up to $60\%$ parallel efficiency.

      
% The Kernel-Free Boundary Integral (KFBI) method has been effectively utilized for solving partial differential equations (PDEs) in irregular domains. Diverging from conventional boundary integral methods, the computation of boundary integrals in KFBI is executed through the resolution of equivalent simplified interface problems on Cartesian grids, utilizing rapid algorithms. This approach eliminates the need for analytical expressions of fundamental solutions or Green's functions in the evaluation of boundary or volume integrals.

% While existing implementations of KFBI methods predominantly utilize CPU platforms, the superior computational capabilities and extensive memory bandwidth of GPU architecture offer an efficient resolution to computational bottlenecks. This paper delineates the algorithms adapted for both single-GPU and multiple-GPUs applications. On a single GPU, individual threads are designated for managing correction, interpolation, and jump calculations. The algorithm is expanded to multiple GPUs to enhance the processing of larger-scale problems. Contrary to the single GPU approach, the arrowhead decomposition method is employed in multi-GPU settings, ensuring optimal computational efficiency and load balancing. Numerical examples demonstrate the proposed algorithm's second-order accuracy and efficiency, with the single-GPU solver achieving speeds 50-200 times faster than traditional CPU implementations. Additionally, the eight-GPU distributed solver attains up to 60% parallel efficiency.
\end{abstract}
%%%%% end %%%%%%%%%%%

%%%%% AMS/PACs/Keywords %%%%%%%%%%%
%\pac{}
\ams{52B10, 65D18, 68U05, 68U07}
\keywords{Arrowhead decomposition method, GPU-accelerated kernel-free boundary integral method, Irregular domains}

%%%% maketitle %%%%%
\maketitle

%%%% Start %%%%%%
\include*{intro}
\include*{KFBI}
\include*{ONEGPU}
\include*{MULTIGPU}
\include*{RESULT}
\include*{Discussion}
%\include*{Appendix}

% \section{Introduction}
% \label{sec1}
% In the past two decades, there has been important progress in developing adaptive mesh methods for PDEs.
% Mesh adaptivity is usually of two types in form: local mesh refinement and moving mesh method.  ....

% \section{Preparation of Manuscript}
% \label{sec2}
% The Title Page should contain the article title, authors' names and complete affiliations,
% and email addresses of all authors. The Abstract should provide a brief summary of the main findings of the paper.

% References should be cited in the text by a number in square brackets.
% Literature cited should appear on a separate page at the end of the article
% and should be styled and punctuated using standard abbreviations for journals
% (see Chemical Abstracts Service Source Index, 1989). For unpublished lectures of symposia,
% include title of paper, name of sponsoring society in full, and date.
% Give titles of unpublished reports with "(unpublished)" following the reference.
% Only articles that have been published or are in press should be included in the references.
% Unpublished results or personal communications should be cited as such in the text.
% Please note the sample at the end of this paper.

% Equations should be typewritten whenever possible and the number placed in parentheses at the right margin.
% Reference to equations should use the form "Eq. (2.1)" or simply "(2.1)." Superscripts and subscripts should
% be typed or handwritten clearly above and below the line, respectively.

% Figures should be in a finished form suitable for publication. Number figures consecutively with Arabic numerals.
% Lettering on drawings should be of professional quality or generated by high-resolution computer graphics and must be
% large enough to withstand appropriate reduction for publication.
% For example, if you use {\sf MATLAB} to do figure plots,
% axis labels should be at least point 18. Title should be 24 points or above. Tick marks labels
% better have 14 points or above. Line width should be 2 (or above).

% Illustrations in color in most cases can be accepted
% only if the authors defray the cost. At the Editor's discretion a limited number of color figures each year of special
% interest will be published at no cost to the author\cite{arrowhead2017}.

%%%% Acknowledgments %%%%%%%%
\section*{Acknowledgments}
This work is financially supported by the Strategic Priority Research Program of Chinese Academy of Sciences(Grant No. XDA25010405). It is also partially  supported by the National Key R\&D Program of China, Project Number 2020YFA0712000, the National Natural Science Foundation of China (Grant No. DMS-11771290) and the Science Challenge Project of China (Grant No. TZ2016002). Additionally, it is supported by the Fundamental Research Funds for the Central Universities. 

%%%% Bibliography  %%%%%%%%%%
% \begin{thebibliography}{99}
% \bibitem{Berger}M. J. Berger and P. Collela, Local adaptive mesh refinement
% for shock hydrodynamics,
% J. Comput. Phys., 82 (1989), 62-84.
% \bibitem{deBoor}C. de Boor,  Good Approximation By Splines With Variable Knots II, in Springer Lecture
%  Notes Series 363, Springer-Verlag, Berlin, 1973.
% \bibitem{TanTZ} Z. J. Tan, T. Tang and Z. R. Zhang, A simple moving mesh method for one- and
% two-dimensional phase-field equations, J. Comput. Appl. Math., to appear.
% \bibitem{Toro}E. F. Toro, Riemann Solvers and Numerical Methods for Fluid Dynamics,
% Springer-Verlag Berlin Heidelbert, 1999.
% \end{thebibliography}

\bibliographystyle{unsrt} 
\bibliography{ref} 
\end{document}
