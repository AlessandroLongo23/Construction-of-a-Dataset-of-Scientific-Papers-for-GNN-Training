\section{Evaluation and Conclusion}\label{Evaluation and Conclusion}

In this paper, we introduced the first
notion of (annotated) dependency pairs
and the first DP framework
for relative termination, which also features suitable
dependency graph and reduction pair processors for relative ADPs.
Of course, 
further classical DP processors can be adapted to our relative ADP framework as well. For
example, in our implementation of the novel ADP framework in our tool \aprove{} \cite{JAR-AProVE2017},
we also included a
straightforward adaption of the classical \emph{rule removal processor}
\cite{gieslLPAR04dpframework}, see 
\Cref{Appendix}.\footnote{This processor works
analogously to the preprocessing
at the beginning
of \Cref{Relative DP Framework}
which can be used to remove duplicating rules: For an  ADP problem $(\C{P},\C{P}^{=})$, it tries to find
a reduction pair 
$(\succsim, \succ)$ 
where $\succ$ is closed under contexts
such that $\flat(\C{P} \cup \C{P}^{=}) \subseteq {\succsim}$.
Then for $\PP_{\succ} \subseteq \C{P} \cup \C{P}^{=}$ 
with $\flat(\C{P}_{\succ}) \subseteq {\succ}$, the processor replaces the ADP by 
$(\C{P} \setminus \PP_{\succ}, \C{P}^{=} \setminus \PP_{\succ})$.}
In future work, we will investigate how to use our new form of 
ADPs for full (instead of innermost) rewriting also in the probabilistic setting 
and for complexity analysis.



To evaluate the new relative ADP framework, we compared its implementation in 
``\emph{new} \aprove{}'' 
to all other tools that participated
in the most recent \emph{termination competition (TermComp 2023)} \cite{termcomp}
on relative rewriting, i.e.,
\natt{} \cite{natt_sys_2014},  \ttttwo{} \cite{ttt2_sys}, \mnm{} \cite{FSCD19}, and ``\emph{old} \aprove{}'' which did
not yet contain the contributions of the current paper.
In \emph{TermComp 2023}, 
98 benchmarks were used for relative termination. However, these benchmarks only consist
of
examples where the main TRS
$\R$ dominates the base TRS $\R^{=}$ (i.e., which can be handled by \Cref{theorem:main-relative-rewrite-corollary-yamada-1} from
\cite{iborra2017relative})
or which can already
be solved via simplification orders directly.
Therefore, we extended the collection by
17 new examples,
including both
$\R_{\tdivl}/\R^{=}_{\tset}$ from \Cref{ex:divlTRS,ex:mset1},
and our leading example $\R_{\tdivl} / \R^{=}_{\tset 2}$ 
from \Cref{ex:mainExample} (where only \emph{new} \aprove{} can prove SN).
Except for $\R_{\tdivl}/\R^{=}_{\tset}$, in these examples
$\R$ does not dominate $\R^{=}$. 
Most of these examples adapt well-known classical TRSs from the
\emph{Termination Problem Data Base} \cite{tpdb} used at \emph{TermComp}
to the relative setting.
In the following table,
the number in the ``YES'' (``NO'') row indicates for how many of the 115
examples the respective
tool could prove (disprove) relative termination and ``MAYBE'' refers to the benchmarks
where the tool could not solve the problem within the timeout of 300~s per example. The numbers in
brackets are the respective results when only considering our new 17 examples.
``AVG(s)'' gives the average runtime of the tool
on solved examples in seconds.

\begin{center}
{\small    \begin{tabular}{||c | c | c | c | c | c||}
     \hline
      & \emph{new} \aprove & \natt & \emph{old} \aprove & \ttttwo & \mnm  \\ [0.5ex] 
     \hline
     YES & 78 (17) & 65 (7) & 47 (4) & 39 (3) & 0 (0)  \\ 
     \hline
     NO & 13 (0)& 5 (0)& 13 (0)& 7 (0)& 13 (0)\\ 
     \hline
     MAYBE & 24 (0)& 45 (10)& 55 (13)& 69 (14) & 102 (17) \\ 
     \hline
      AVG(s) & 6.68 & 0.38 & 3.67 & 1.61 & 1.28  \\ 
     \hline
    \end{tabular}}
\end{center}



The table clearly shows that while \emph{old} \aprove{} was already the second most powerful
tool for relative termination, the integration of the ADP framework in \emph{new} \aprove{}
yields a substantial advance in power (i.e., it only fails on 24 of the examples, compared
to 45 and 55 failures of \natt{} and \emph{old} \aprove, respectively).
In particular,  previous tools (including 
\emph{old} \aprove{}) often have problems with
relative TRSs where the main TRS does
 not dominate the base TRS, whereas the ADP framework can handle
 such examples.


A special form of relative TRSs are \emph{relative string rewrite systems (SRSs)}, where all function
symbols have arity 1.
Due to the base ADPs with two annotated
symbols on the right-hand side, 
here the ADP framework is less powerful
than dedicated techniques for string rewriting.
For the 403 relative SRSs at \emph{TermComp 2023},
the ADP framework only finds 71 proofs, mostly due to the dependency graph and the rule removal processor, 
while termination analysis via \aprove's
standard strategy for relative SRSs
succeeds on 209 examples, and the two most powerful tools for relative SRSs at
\emph{TermComp 2023} (\mnm{} and \matchbox{} \cite{matchbox})
succeed on 274 and 269 examples, respectively.


Another special form of relative rewriting is \emph{equational rewriting}, where one has
a set of equations $E$ which correspond to relative rules that can be applied in both directions.
In \cite{RTA01}, DPs were adapted to equational rewriting.
However, this approach requires
$E$-unification to be decidable and finitary
(i.e., for (certain) pairs of terms, 
it has to compute  finite complete sets of $E$-unifiers).
This works well if $E$
are AC- or C-axioms, and for this special case, dedicated techniques like
\cite{RTA01} are more powerful than our new ADP framework for relative termination.
For example, on the 76 AC- and C-benchmarks  for equational rewriting at \emph{TermComp 2023},
the  relative ADP framework  finds 36 proofs, while dedicated
tools for AC-rewriting like \aprove's equational strategy or \muterm{} \cite{gutierrez_mu-term_2020} succeed on
66 and 64 examples, respectively.
However, in general, the requirement of a finitary
$E$-unification algorithm is a  hard restriction.
In contrast to existing tools for equational rewriting,
our new ADP framework can be used for arbitrary
(non-duplicating) relative rules.

For details on our experiments, our collection of examples,
and for instructions on how to run our implementation
in \textsf{AProVE} via its \emph{web interface} or locally, see
\url{https://aprove-developers.github.io/RelativeDTFramework/}
