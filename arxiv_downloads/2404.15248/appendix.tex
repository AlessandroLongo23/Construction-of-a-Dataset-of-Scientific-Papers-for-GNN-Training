\section{Appendix}\label{Appendix}
In this appendix, we give all proofs for our new results and observations, and present an
additional rule removal processor for our ADP framework (\Cref{thm:RelRRP}) that results
from a straightforward adaption of the corresponding processor from the classical DP
framework \cite{gieslLPAR04dpframework}.

Before we start with the proof of the chain criterion,
for any infinite rewrite sequence $t_0 \to_{\R\cup \R^{=}} t_1 \to_{\R\cup \R^{=}} \ldots$
we define \emph{origin graphs} 
that indicate
which subterm of $t_{i+1}$ ``originates'' from which subterm of $t_i$. Therefore, these
graphs also indicate how annotations can move in the chains that correspond to this
rewrite sequence.
Note that due to the choice of the VRF and due to the fact that at most two defined symbols
are annotated in the right-hand sides of ADPs,
there are multiple possibilities 
for the annotations to move in a chain.
Each origin graph corresponds to one possible way that the annotations can move.

\begin{definition}[Origin Graph]\label{def:orig}
    Let $\R$ and $\R^{=}$ be two TRSs 
    and let $\Theta: t_0 \to_{\R \cup \R^{=}} t_1 \to_{\R \cup \R^{=}} \ldots$ be a rewrite sequence.
    A graph with the nodes $(i,\pi)$ for all $i \in \IN$ and all $\pi \in \pos(t_i)$
    is called an \defemph{origin graph} for $\Theta$
    if the edges are defined as follows:
    For $i \in \IN$,
    let
the rewrite step  $t_i \to_{\R \cup \R^{=}} t_{i+1}$ be performed using the rule
$\ell \to r \in \R\cup \R^{=}$, the position $\tau$, and the substitution $\sigma$,
i.e., $t_i|_\tau = \ell \sigma$ and $t_{i+1} = t_i[r \sigma]_\tau$.
    Furthermore, let $\pi \in \pos(t_i)$.
    \begin{itemize}
        \item[(a)] If $\pi < \tau$ or $\pi \bot \tau$ (i.e., $\pi$ is above or parallel to $\tau$),        
          then there is an edge from $(i,\pi)$ to $(i+1,\pi)$. The reason is that
          if position $\pi$ is annotated in $t_i$,
        then in a chain it would remain annotated in $t_{i+1}$.
      \item[(b)] For $\pi=\tau$,
        there are at most two outgoing edges from $(i,\pi)$ if
        the rule is in $\R^{=}$ and at most one edge if the rule is in $\R$. These edges
        lead to nodes of the
        form
        $(i+1,\pi.\alpha)$ for $\alpha \in \pos_{\Sigma}(r)$.
        The reason is that rules in $\ADPairMain{\R}$ contain at most one annotation and
        rules in $\ADPairBase{\R}$ contain at most two annotations in their right-hand sides. Moreover, 
        if $\pi$ is annotated in $t_{i}$, then we may create annotations
        at the positions $\pi.\alpha$ in the right-hand side 
        of the used ADP.
      \item[(c)] For every variable position
$\alpha_\ell \in \pos_\VSet(\ell)$, either there are no outgoing edges from any node of
        the form $(i,\tau.\alpha_\ell.\beta)$ with
 $\beta \in \NN^*$, or there is a 
        position $\alpha_r \in \pos_{\VSet}(r)$ with
        $r|_{\alpha_r} = \ell|_{\alpha_\ell}$ and for all $\beta \in \NN^*$ with
        $\alpha_\ell.\beta \in \pos(\ell\sigma)$, there is an edge from 
        $(i,\tau.\alpha_\ell.\beta)$ to $(i+1,\tau.\alpha_r.\beta)$. This captures the
        behavior of VRFs.
        \item[(d)] For all other positions $\pi \in \pos(t_i)$, there is no outgoing edge from the node $(i, \pi)$.
    \end{itemize}
\end{definition}

Due to \Cref{def:orig}, we have the following connection between origin graphs and
chains. For every origin graph for a
rewrite sequence $t_0 \to_{\R \cup \R^{=}} t_1 \to_{\R \cup \R^{=}} \ldots$ and every
$\tilde{t}_0$ such that $\flat(\tilde{t}_0) = t_0$, there is a chain
 $\tilde{t}_0 \tored{}{}{\ADPairMain{\R} \cup \ADPairBase{\R^{=}}} \tilde{t}_{1}
\tored{}{}{\ADPairMain{\R} \cup \ADPairBase{\R^{=}}} \ldots$ with
$\flat(\tilde{t}_i) = t_i$ for all $i \in \NN$ such that:
\[ \begin{array}{c}
\text{$\pi \in 
\posT(\tilde{t}_i)$}\\
\text{iff}\\
\text{there is a path in the origin graph
from $(0,\tau)$ to $(i,\pi)$ for some $\tau
\in \posT(\tilde{t}_0)$}
  \end{array}\]

\begin{example}\label{graph1}
    Let $\R \cup \R^{=}$ have 
    the rules $\ta(x) \to \tb(x)$, $\tg(x,x) \to \ta(x)$,   $\tb(x) \to \tb(x)$, and
    consider
    the rewrite sequence
    $\Theta:\tg(\underline{\ta(\O)}, \tb(\O)) \to_{\R\cup \R^{=}} \underline{\tg(\tb(\O), \tb(\O))} \to_{\R\cup \R^{=}} \ta(\tb(\O)) \to_{\R\cup \R^{=}} \ldots$. So here we
    have $t_0 = \tg(\ta(\O), \tb(\O))$, $t_1 = \tg(\tb(\O), \tb(\O))$, and $t_2 =
    \ta(\tb(\O))$.
    The following is a possible origin graph for $\Theta$.
    \[ \xymatrix @-1.5pc {
        t_0=&\tg \ar@{-}[d] &(& \ta \ar@{-}[d]&(&\O\ar@{-}[d]&), & \tb\ar@{-}[d] & (&\O\ar@{-}[d]&)) \\
        t_1=&\tg \ar@{-}[d] &(    & \tb \ar@{-}[d]&(&\O\ar@{-}[d]&), & \tb\ar@{-}[dllll] & (&\O\ar@{-}[dllll]&)) \\
        t_2=&\ta            &( & \tb           &(&\O          &)) &                   &  &  &
    }
    \]
\end{example}

We can now prove the chain criterion in the relative setting.
In the following,
we often use the notation $\pos_{f}(t)$ instead of $\pos_{\{f\}}(t)$
for a term $t$ and a single symbol or variable $f \in \Sigma \cup \VSet$.

\RelChainCriterion*

\begin{myproof}
    \underline{Completeness}:
    Assume that the ADP problem $(\ADPairMain{\R}, \ADPairBase{\R^{=}})$ is not SN.
    Then, there exists an infinite $(\ADPairMain{\R}, \ADPairBase{\R^{=}})$-chain,
    i.e., an infinite rewrite sequence $t_0 \tored{}{}{\ADPairMain{\R} \cup \ADPairBase{\R^{=}}} t_{1} \tored{}{}{\ADPairMain{\R} \cup \ADPairBase{\R^{=}}} \ldots$
    that uses an infinite number of rewrite steps with $\ADPairMain{\R}$ and Case $(\mathbf{pr})$.
    
    By removing all annotations we obtain the rewrite sequence
    $\flat(t_0) \to_{\R \cup \R^{=}} \flat(t_1) \to_{\R \cup \R^{=}} \ldots$
    that uses an infinite number of rewrite steps with $\R$.
    Hence, $\R / \R^{=}$ is not SN either.

    \smallskip

    \noindent
    \underline{Soundness}:
    Assume that $\R / \R^{=}$ is not SN.
    Then there exists an infinite sequence $\Theta: t_0 \to_{\R \cup \R^{=}} t_1 \to_{\R \cup \R^{=}} \ldots$ 
    that uses an infinite number of $\R$-rewrite steps.

    We will define a sequence $\tilde{t}_0, \tilde{t}_1, \ldots$ of annotated terms 
    such that $\flat(\tilde{t}_i) = t_i$ and
    $\tilde{t}_i \tored{}{}{\ADPairMain{\R} \cup \ADPairBase{\R^{=}}} \tilde{t}_{i+1}$ for all
    $i \in \IN$, 
    where we use an infinite number of rewrite steps with $\ADPairMain{\R}$ and Case $(\mathbf{pr})$.
    This is an infinite $(\ADPairMain{\R}, \ADPairBase{\R^{=}})$-chain, and hence, $(\ADPairMain{\R}, \ADPairBase{\R^{=}})$ is not SN either.

    W.l.o.g., let $t_0$ be a minimal term that is
    non-terminating w.r.t.\ $\R/\R^{=}$, i.e., there exists no proper subterm of $t_0$ that starts 
    a rewrite sequence which uses an infinite number of $\R$-steps.
    Such a minimal term exists, since there are only finitely many subterms of $t_0$.
    Next, we prove that there exists an origin graph for $\Theta$ with a path from
    $(0,\varepsilon)$ to some node $(i,\pi)$
    and a path  from  $(0,\varepsilon)$ to some node  $(i+1,\pi')$ with $i \in \IN$ such that
    the rewrite step $t_i \to_\R t_{i+1}$ takes place at position $\pi$, and
    $t_{i+1}|_{\pi'}$ is a minimal non-terminating term w.r.t.\ $\R/\R^{=}$.
    Then, according to the definition of the origin graph, there exists a chain
    $\tilde{t}_0, \ldots, \tilde{t}_i,
    \tilde{t}_{i+1}$ with $\flat(\tilde{t}_j) = t_j$ for all $1 \leq j \leq i+1$
    such that the positions $\pi$ in $\tilde{t}_i$ and $\pi'$ in $\tilde{t}_{i+1}$ are annotated.
    Hence, the  rewrite step
 $\tilde{t}_i \tored{}{}{\ADPairMain{\R} \cup \ADPairBase{\R^{=}}} \tilde{t}_{i+1}$ 
    is performed with $\ADPairMain{\R}$ and Case $(\mathbf{pr})$.
    Furthermore, there is a minimal non-terminating subterm of $\tilde{t}_{i+1}$ whose root is annotated.
    Thus, we can perform the whole construction again to create another chain
    starting in $\tilde{t}_{i+1}$
    that ends in a rewrite step with $\ADPairMain{\R}$ and Case $(\mathbf{pr})$.
    By repeating this construction infinitely often, we generate our desired infinite chain.



    It remains to prove that such an origin graph exists for
    every minimal non-terminating term $t_0$.
Let $k$ be the arity of 
$t_0$'s root symbol.
Then for $i \in \NN$, by induction we  now define
$t_i^1,\ldots,t_i^k$ such that $t_i^1,\ldots,t_i^k \triangleleft t_i$ at parallel
positions and such that $t_0^j \to_{\R \cup \R^=}^{\leq 1} t_1^j \to_{\R \cup \R^=}^{\leq
  1} \ldots$ for all $1 \leq j \leq k$, where $\to_{\R \cup \R^=}^{\leq 1} = (\to_{\R \cup \R^=} \cup =)$, i.e., 
$\to_{\R \cup \R^=}^{\leq 1}$ is the reflexive closure of $\to_{\R \cup \R^=}$. In addition, for $i \in \NN$ we define 
a non-empty context $C_i$
such that $t_i = C_i[q_{1,1},\ldots,q_{1,h_1}, \ldots, q_{k,1}, \ldots, q_{k, h_k}]$ for
subterms $q_{j,1},\ldots,q_{j,h_j} \trianglelefteq t_i^j$ at parallel positions for all $1
\leq j \leq k$.
Moreover, if $i > 0$, then for all pairs of positions $\pi_1, \pi_2 \in \pos_\Sigma(C_i)$
    we show that one
    can construct an origin graph with paths from $(0,\varepsilon)$ to $(i, \pi_1)$ and
from $(0,\varepsilon)$ to
    $(i, \pi_2)$.

  Note that there exists an $i \in \NN$ such that the rewrite
    step from $t_i$ to $t_{i+1}$ is done with an $\R$-rule at a position in $C_i$. 
    The reason is that otherwise,
infinitely many $\R$-steps would be applied on terms of the form $q_{j,b}$. But this would
mean that there exists a $t_0^j$ such that the sequence 
 $t_0^j \to_{\R \cup \R^=}^{\leq 1} t_1^j \to_{\R \cup \R^=}^{\leq
  1} \ldots$ has infinitely many $\R$-steps. However, this 
would be a contradiction to the minimality of $t_0$ because
 $t_0^j$ is a proper subterm of $t_0$.
So we define $t_i^1,\ldots,t_i^k$ and $C_i$ for all $i \geq 0$ until we reach the first $i$
where
an $\R$-step is performed with a redex at a position in $C_i$.



    We start with the case $i=0$. 
Let
    $t_0^{1}, \ldots, t_0^{k}$ be the subterms of $t_0$ at positions $1, \ldots, k$
    (i.e., $t_0^j = t_0|_j$ for all $1 \leq j \leq k$).
    Then, we have $t_0 = C_0[t_0^{1}, \ldots, t_0^{k}]$ for the context $C_0$ that only consists
    of $t_0$'s root symbol applied to $k$ holes.




    In the induction step, we have
    $t_i \to_{\R \cup \R^{=}} t_{i+1}$ using a position $\pi$, a substitution $\sigma$, and a
    rule $\ell \to r$ with $t_i|_{\pi} = \ell \sigma$ and $t_{i+1} = t_i[r
    \sigma]_{\pi}$. Here, we have two cases:
    \begin{itemize}
        \item If $\pi \notin \pos(C_i) \setminus \pos_{\Box}(C_i)$, 
          then $\pi$ must be in some $q_{j,b} \trianglelefteq t_{i}^{j}$.
          So there is an $\alpha \in \pos_{\Box}(C_i)$
such that $\pi =  \alpha.\beta$ for some  $\beta \in \IN^*$.
Hence, we have $t_i|_\pi = 
C_i[q_{1,1},\ldots,q_{1,h_1}, \ldots, q_{k,1}, \ldots, q_{k, h_k}]|_\pi = q_{j,b}|_\beta$
for some $q_{j,b} = t_i^j|_\gamma$. 
     Here, we simply perform the rewrite step on this term, and the context and the other terms remain the same.
        Hence, we have $C_{i+1} = C_{i}$, $t_{i+1}^{j} = t_i^{j}[r \sigma]_{\gamma.\beta}$ and
        $t_{i+1}^{j'} = t_i^{j'}$ for all $1 \leq j \leq k$ with $j' \neq j$.
        Using the subterms $q'_{1,1}, \ldots, q'_{1, h_1'}, \ldots,
        q'_{k,1}, \ldots, q'_{k, h_k'}$ with $q'_{j,b} = q_{j,b}[r \sigma]_{\beta}$ and
        %JG You wrote $\gamma$, but here it should be $\beta$.
        $q'_{c,d} = q_{c,d}$ for all $1 \leq c \leq k$ and $1 \leq d \leq h_c$
        %JG You wrote $h_a$, but it should be $h_c$.
        with $(c,d) \neq (j,b)$,
        we finally get $t_{i+1} = C_{i+1}[q'_{1,1}, \ldots, q'_{1, h_1'}, \ldots,
        q'_{k,1}, \ldots, q'_{k, h_k'}]$, as desired.
        
        It remains to prove that our claim on the paths in the origin graph is still satisfied.
        For all $\tau_1, \tau_2 \in \pos_\Sigma(C_{i+1}) = \pos_\Sigma(C_{i})$, by the induction
        hypothesis there
        exists an origin graph with
        paths from $(0,\varepsilon)$ to $(i, \tau_1)$ and from
$(0,\varepsilon)$ to $(i, \tau_2)$. Since the origin graph has  edges from
        $(i, \tau_1)$ to $(i+1,\tau_1)$ and from
        $(i, \tau_2)$ to $(i+1,\tau_2)$
        by \Cref{def:orig} since $\tau_1$ and $\tau_2$ are above or parallel to $\pi$, 
        there are also paths from $(0,\varepsilon)$ to $(i+1, \tau_1)$ and from
        $(0,\varepsilon)$ to $(i+1, \tau_2)$.
        % Furthermore, by the induction hypothesis
        % there exists a subterm of $t_{i}$ that is minimally non-terminating, and by the
        % minimality of $t_0$, this subterm is at a position in $C_i$.
        % Hence, there is also a subterm of $t_{i+1}$ at a position $\pi'$ in
        % $C_{i+1}$
        % that is minimally non-terminating and thus, there is also a path from $(0,\varepsilon)$ to
        % $(i+1, \pi')$.\oldcomment{JG Here, there is something wrong. In the induction step we would
        % have to prove this for $(i+2,\pi')$.}

        \item Now we consider the case $\pi \in \pos(C_i) \setminus \pos_{\Box}(C_i)$.
        If the step $t_i \to_{\R \cup \R^{=}} t_{i+1}$ is an $\R$-step, then we stop, because
        we reached the first $\R$-step where the redex is at a position in $C_i$.

        So we now have $t_i \to_{ \R^{=}} t_{i+1}$. We define $t_{i+1}^j = t_i^j$ for all
        $1 \leq j \leq k$, but we still need to define the context and the subterms for
        each $t_{i+1}^j$. We will now define a suitable VRF
        $\varphi:\pos_{\VSet}(\ell)\to \pos_{\VSet}(r) \cup \{ \bot \}$ step by step, where we initialize
        $\varphi$ to yield $\bot$ for all arguments.
        %% Let $\Delta = 
        %% \{\rho_1, \ldots, \rho_w\} \subseteq \pos_{\VSet}(\ell)$ be the set of all positions
        %% of variables in $\ell$ such that $\rho_z \in \pos(C_i)\setminus \pos_{\Box}(C_i)$
        %% for all $1 \leq z \leq
        %% w$, i.e., the positions
        %% of these variables are still in the context $C_i$.
        %% All positions from $\pos_{\VSet}(\ell) \setminus \Delta$ are outside the context
        %% (or holes $\Box$).
        %%
        For every $\rho \in \pos_{\VSet}(\ell)$,
        % JCK Nevermind, this should work.
        % \comment{JG I simplified this, because
        %   $\Delta$ was no longer needed to define $\varphi$.\\ JCK We have to prioritize $\Delta$ before we move to
        %   the other positions. Hence, we need to introduce it again.}
        if possible,
        we let
        $\varphi(\rho) \in \pos_{\VSet}(r)$ be a position of $r$ that is not yet in the
        image of $\varphi$ and where
        $\ell|_{\rho} = r|_{\varphi(\rho)}$. If there is no such position
        of $r$, then we keep $\varphi(\rho) = \bot$.
        %% Here, if possible,
        %% we choose $\varphi(\rho_z)$ to be a position from $\pos_{\VSet}(r)$ that was not yet
        %% in the image of $\varphi$, otherwise, we keep $\varphi(\rho_z) = \bot$.
        %% Afterwards, we do the same for all $\alpha_\ell \in \pos_{\VSet}(\ell) \setminus \Delta$. 
        %% Let $\varphi(\alpha_\ell) \in \pos_{\VSet}(r)$ be a position 
        %% with $\ell|_{\alpha_\ell} = r|_{\varphi(\alpha_\ell)}$. Again, if possible,
        %% we choose $\varphi(\alpha_\ell)$ to be a position from $\pos_{\VSet}(r)$ that was not yet
        %% in the image of $\varphi$, otherwise, we keep $\varphi(\rho_z) = \bot$.
 
        Let $\varphi|^{\pos_{\VSet}(r)}$
denote the restriction of $\varphi$ to the codomain $\pos_{\VSet}(r)$ (i.e.,
$\varphi|^{\pos_{\VSet}(r)}$
is only defined on those $\rho \in \pos_{\VSet}(\ell)$ where
$\varphi(\rho) \neq \bot$). Then 
 $\varphi|^{\pos_{\VSet}(r)}$ is surjective, since
        rules of $\R^=$ must not be duplicating, and injective, since we only extend the
        function $\varphi$ if a position of a variable in $r$ was not already in the image
        of $\varphi$.
        %JG I changed this again, because the notation $\pi.(\pos(C_i)\setminus
        %   \pos_{\Box}(C_i))$ is not standard and we would have to explain it.
        Let $\{\rho_1, \ldots, \rho_w\} \subseteq \pos_{\VSet}(\ell)$
        be those positions
        of variables from $\ell$  in the context $C_i$ that are no holes
        (i.e., where $\pi.\rho_z \in
        \pos(C_i)\setminus
        \pos_{\Box}(C_i)$)
        and where 
        $\varphi(\rho_z) \neq \bot$ for all $1 \leq z \leq w$.
        Then we define the new context $C_{i+1}$ as 
        $C_{i+1} = C_i[r \delta_{\Box}]_{\pi}[C_i|_{\pi.\rho_1}]_{\pi.\varphi(\rho_1)} \ldots [C_i|_{\pi.\rho_{w}}]_{\pi.\varphi(\rho_{w'})}$ 
        using the substitution $\delta_{\Box}$ that maps every variable to $\Box$.
        So $C_{i+1}$ results from $C_i$ by replacing the subterm at position $\pi$ by $r$
        where all variables are substituted with $\Box$, and then restoring
        the part of the context that was inside the substitution.
        
       
        Next, we need to define the subterms
        $q'_{j,1}, \ldots, q'_{j, h_j'}$ of $t_{i+1}^j$ 
        such that $t_{i+1} = C_{i+1}[q'_{1,1}, \ldots, q'_{1, h_1'}, \ldots,
          q'_{k,1}, \ldots, q'_{k, h_k'}]$.
        Let $\kappa$ be the position of some $q_{j,b}$ in $t_i$.
        First, consider the case that
        for some $\alpha_{\ell} \in \pos_{\VSet}(\ell)$ and $\beta \in \IN^*$ we have $\pi.\alpha_{\ell}.\beta = \kappa$, 
        i.e., $q_{j,b}$ is completely inside the substitution.
        If $\varphi(\alpha_{\ell}) = \bot$, then we remove the subterm $q_{j,b}$ in $t_{i+1}$.
        Otherwise, $q_{j,b}$ is one of the subterms $q'_{j,b'}$ and it moves from position $\pi.\alpha_{\ell}.\beta$
        in $t_i$ to position $\pi.\varphi(\alpha_{\ell}).\beta$ in $t_{i+1}$.
        Note that there are no two such $q_{j_1,b_1}$, $q_{j_2,b_2}$ that move to the same position,
        due to injectivity of $\varphi$.
        Second, $q_{j,b}$ is also one of the subterms $q'_{j,b'}$ if the
        position $\kappa$ is parallel to $\pi$.
        Here, the position of $q_{j,b}$ in $t_{i+1}$ remains the same as in $t_i$.
        Third, we consider the case that 
        there exist some $\alpha_{\ell} \in \pos_{\VSet}(\ell)$ such that $\kappa < \pi.\alpha_{\ell}$,
        i.e., $q_{j,b}$ is a subterm of the redex but not completely inside the substitution.
        For all such $\alpha_{\ell}$, let $\chi_{\alpha_\ell} \in \IN^*$ such that
        $\kappa.\chi_{\alpha_\ell} = \pi.\alpha_{\ell}$.
        Instead of the subterm $q_{j,b}$, we now 
        use the subterms $q_{j,b}|_{\chi_{\alpha_\ell}}$ for all those $\alpha_\ell$ with $\varphi(\alpha_{\ell}) \neq \bot$. 
        Now $q_{j,b}|_{\chi_{\alpha_\ell}}$ is a subterm of $t_{i+1}$ at position $\pi.\varphi(\alpha_{\ell})$.
        All in all, we get $t_{i+1} = C_{i+1}[q'_{1,1}, \ldots, q'_{1, h_1'}, \ldots,
        q'_{k,1}, \ldots, q'_{k, h_k'}]$ and $q'_{j,1},\ldots,q'_{j,h_j'} \trianglelefteq t_{i+1}^j$ at parallel positions for all $1
        \leq j \leq k$.

        Finally, we prove that our claim on the paths in the origin graph is
        still satisfied.
        Consider two positions $\tau_1, \tau_2 \in \pos_\Sigma(C_{i+1})$, and let $\tau \in \{\tau_1, \tau_2\}$.
        If $\tau$ is above or parallel to $\pi$, then by the induction
        hypothesis there exists an origin graph with
        a path from $(0,\varepsilon)$ to $(i, \tau)$. Since the origin graph has an edge from
        $(i, \tau)$ to $(i+1,\tau)$ by \Cref{def:orig}, 
        there is a path from $(0,\varepsilon)$ to $(i+1, \tau)$.
        If $\tau$ has the form $\pi.\alpha$ with $\alpha \in \pos_\Sigma(r)$ in $C_{i+1}$,
        then by the induction
        hypothesis there is an origin graph with
        a path from $(0,\varepsilon)$ to $(i, \pi)$. Since the origin graph can be chosen to
        have an edge from
        $(i, \pi)$ to $(i+1,\pi.\alpha)$ by \Cref{def:orig}, 
        there is a path from $(0,\varepsilon)$ to $(i+1,\pi.\alpha)$.
        Note that if both $\tau_1$ and $\tau_2$ have such a form (i.e., $\tau_1 =
        \pi.\alpha_1$ and $\tau_2 = \pi.\alpha_2$ with $\alpha_1, \alpha_2 \in
        \pos_\Sigma(r)$),
        then the origin graph can be chosen to have edges from $(i,\pi)$ to both 
        $(i+1,\pi.\alpha_1)$ and $(i+1,\pi.\alpha_2)$, as we can have  two such edges for
        a rewrite step with a relative rule from $\R^=$.
        Finally, if $\tau$ has the form $\pi.\alpha_r.\beta$ with
        $\alpha_r \in \pos_\VV(r)$ in $C_{i+1}$,
        then since the image of $\varphi$ consists of all variable positions from
         $\pos_\VV(r)$ (due to non-duplication of $\R^=$),
        there is a $\rho \in \pos_\VV(\ell)$ with
        $\ell|_{\rho} = r|_{\varphi(\rho)}$ where $\varphi(\rho)
        = \alpha_r$.
        By the induction hypothesis, there is an origin graph with
        a path from $(0,\varepsilon)$ to $(i, \pi.\rho.\beta)$ and by 
        \Cref{def:orig}, the origin graph can be chosen to have an edge from
        $(i, \pi.\rho.\beta)$ to $(i+1, \pi.\alpha_r.\beta)$.
        All in all, there exists an origin graph with paths from $(0,\varepsilon)$ to $(i, \tau_1)$
        and from $(0,\varepsilon)$ to $(i, \tau_2)$.
    \end{itemize}


So we have shown that 
    there exists an origin graph for $\Theta$ with a path from
    $(0,\varepsilon)$ to some node $(i,\pi)$
    such that
    the rewrite step $t_i \to_\R t_{i+1}$ takes place at position $\pi$ in the context
    $C_i$. Moreover, for any further position $\pi' \in \pos_\Sigma(C_i)$, in this origin
    graph there is also a path from   $(0,\varepsilon)$ to some node $(i,\pi')$.
In a similar way, one can also extend the construction one step further to define 
$t_{i+1}^1,\ldots,t_{i+1}^k$ and a non-empty context $C_{i+1}$ such that
 $t_{i+1} = C_{i+1}[q_{1,1},\ldots,q_{1,h_1}, \ldots, q_{k,1}, \ldots, q_{k, h_k}]$ for
subterms $q_{j,1},\ldots,q_{j,h_j} \trianglelefteq t_{i+1}^j$ (not necessarily at parallel positions, 
since the used $\R$-step may be duplicating).
%  JCK The statement below may not true anymore, as this last step may be duplicating. 
%  But it does not matter as the requirement of being parallel is not needed for the last step. 
%  It was only needed for guaranteeing that we have an $\R$-redex in the context, which is already
%  satisfied.
%
%  at parallel positions for all $1 \leq j \leq k$
Moreover,  for all  positions $\pi_1 \in \pos_\Sigma(C_i)$ and  $\pi_2 \in \pos_\Sigma(C_{i+1})$
   one
    can construct an origin graph with paths from $(0,\varepsilon)$ to $(i, \pi_1)$ and
from $(0,\varepsilon)$ to
    $(i+1, \pi_2)$.
For this last step,
 the cases of the proof are exactly the same, the only difference is that from node $(i,\pi)$,
    there may only be a single outgoing edge (since the used rule was in $\R$).
    But as we are only looking for a single position $\pi_2$ in $\pos_\Sigma(C_{i+1})$
    (and no pairs of positions anymore), 
    this suffices for the construction.

Then we have shown that the desired origin graph exists, by choosing $\pi_1$ to be $\pi$
(the position of the redex in the $\R$-step from $t_i$ to $t_{i+1}$)
and by choosing $\pi_2$ to be the position of a minimal non-terminating subterm of
$t_{i+1}$. Note that this position must be in $\pos_\Sigma(C_{i+1})$, because all subterms
$q_{j,b}$ are terminating w.r.t.\ $\R/\R^=$.    
\end{myproof}


For the following proof,
recall that $t \trianglelefteq_{\#}^{\tau} s$ if
$\tau \in \posT(s)$ and $t = \flat(s|_\tau)$.

\DerelProcOne*

\begin{myproof}
Completeness of any processor that yields the empty set is trivial. So we only have to
consider soundness.
  
    \underline{Only if}:
    Assume that $(\DP{\C{P}},\flat(\C{P} \cup \C{P}^{=}))$ is not SN.
    Then
there exists an infinite sequence $t_0, t_1, t_2, \ldots$ 
    with $t_i \rootto_{\DP{\C{P}}} \circ \to_{\flat(\C{P} \cup \C{P}^{=})}^* t_{i+1}$ for all $i \in \IN$.
    We now create a sequence $s_0, s_1, \ldots$ of annotated terms such that 
    $s_i \tored{}{(\mathbf{pr})}{\C{P}} \circ \tored{}{(\mathbf{r}) \, *}{\C{P} \cup
      \C{P}^{=}} s_{i+1}$
and $\flat(t_i) \trianglelefteq_{\#} s_i$    
    for all $i \in \IN$, which  by \Cref{theorem:relative-chain-crit} implies that $(\C{P},\C{P}^{=})$ is not SN, i.e., the
    processor is not sound.
    Initially, we start with the term $s_0 = t_0$.
    We have 
    \[
        t_0 \rootto_{\DP{\C{P}}} t_{0,1} \to_{\flat(\C{P} \cup \C{P}^{=})} t_{0,2} \to_{\flat(\C{P} \cup \C{P}^{=})} \ldots \to_{\flat(\C{P} \cup \C{P}^{=})} t_{1}
    \]
    In the first rewrite step, there is a DP $\ell^\# \to t^\# \in \DP{\C{P}}$ and a substitution $\sigma$
    such that $t_0 = \ell^\# \sigma$ and $t_{0,1} = t^\# \sigma$.
    This DP $\ell^\# \to t^\#$ results from some ADP $\ell \to r \in \C{P}$ with
    $t \trianglelefteq_{\#}^{\tau} r$
    for some position $\tau \in \pos(r)$.
    We can rewrite $s_0$ with $\ell \to r$ and the substitution $\sigma$ at the root
    resulting in $s_{0,1} = r\sigma$ with
    $\flat(t_{0,1}) \trianglelefteq_{\#}^{\tau}
    r\sigma = s_{0,1}$. 
    Then, we mirror each step that takes place at position $\pi$ in $t_{0,i}$ at position $\tau.\pi$ in $s_{0,i}$.
    To be precise, if we have $t_{0,i+1} = t_{0,i}[r' \sigma']_{\pi}$ using a rule $\ell' \to \flat(r') \in \flat(\C{P} \cup \C{P}^{=})$, the substitution $\sigma'$, and the position $\pi$, then we have $\flat(t_{0,i}) \trianglelefteq_{\#}^{\tau} s_{0,i}$, and can rewrite $s_{0,i}$
    with the ADP $\ell' \to r'$, the substitution $\sigma'$, and the position $\tau.\pi$.
    We get $\flat(s_{0,i+1}) = \flat(s_{0,i}[r' \sigma']_{\tau.\pi})$
    with $\flat(t_{0,i+1}) = \flat(t_{0,i}[r' \sigma']_{\pi}) \trianglelefteq_{\#}^{\tau}  s_{0,i+1}$.
    In the end, we have $\flat(t_1) \trianglelefteq_{\#}^{\tau} s_1$.

    We can now repeat this for each $i \in \IN$ and result in our desired chain.
    
    \smallskip

    \noindent
    \underline{If}:
    Assume that the processor is not sound and that $(\C{P},\C{P}^{=})$ is not SN.
    Then,  by \Cref{theorem:relative-chain-crit} there exists an infinite sequence $t_0, t_1, t_2, \ldots$ 
    of annotated terms such that 
    $t_i \tored{}{(\mathbf{pr})}{\C{P}} \circ \tored{}{*}{\C{P} \cup \C{P}^{=}} t_{i+1}$ 
    for all $i \in \IN$.
    W.l.o.g., let $t_0$ contain only a single annotation (we only need the annotation for the position that leads to infinitely many $\C{P}$-steps with Case $(\mathbf{pr})$).
    We now create a sequence $s_0, s_1, \ldots$ such that 
    $s_i \rootto_{\DP{\C{P}}} \circ \to_{\flat(\C{P} \cup \C{P}^{=})}^* s_{i+1}$
    and $\flat(s_i) \trianglelefteq_{\#} t_i$
    for all $i \in \IN$, 
    which implies that $(\DP{\C{P}},\flat(\C{P} \cup \C{P}^{=}))$ is not SN.
    Due to the condition $\flat(\C{P}^=) = \C{P}^=$, we have 
    \[  
        t_0 \tored{}{(\mathbf{pr})}{\C{P}} t_{0,1} \tored{}{(\mathbf{r})}{\C{P} \cup \C{P}^{=}}  t_{0,2} \tored{}{(\mathbf{r})}{\C{P} \cup \C{P}^{=}} \ldots \tored{}{(\mathbf{r})}{\C{P} \cup \C{P}^{=}} t_{1}
    \]
    The reason that we have no $\tored{}{(\mathbf{pr})}{\C{P}^{=}}$-steps is that
all terms $t_i$ and $t_{i,j}$ 
only contain a single annotation.
  Since $\flat(\C{P}^=) = \C{P}^=$, we only have annotations in $\C{P}$, where every
  right-hand side of a rule only has at most a single annotation.
  Thus, no term $t_i$ or $t_{i,j}$ can have more than one annotation.
    If we would rewrite at the position of this annotation with a rule without
    annotations (e.g., from $\C{P}^{=}$), then we would not have any annotation left 
    and there would be no future $\tored{}{(\mathbf{pr})}{\C{P}}$-step possible anymore.

    In the first rewrite step, there is an ADP $\ell \to r \in \C{P}$, a substitution $\sigma$, and a position $\pi$
    such that $t_0|_{\pi} = \ell^\# \sigma$ and $t_{0,1} = t_{0}[\anno_{\rho}(r \sigma)]_{\pi}$
    for some $\rho \in \pos(r)$.
    Initially, we start with the term $s_0 = \ell^\# \sigma$.
    We can rewrite $s_0$ with $\ell^\# \to t^\# \in \DP{\C{P}}$ for
    $t \trianglelefteq_{\#}^{\rho} r$ and the substitution $\sigma$
    resulting in $s_{0,1} = t^\# \sigma$
    with $\flat(s_{0,1}) = t\sigma  \trianglelefteq_{\#}^{\rho}
    \anno_{\rho}(r \sigma)$, i.e.,
    $\flat(s_{0,1}) \trianglelefteq_{\#}^{\kappa}
    t_{0}[\anno_{\rho}(r \sigma)]_{\pi}
     = t_{0,1}$ for $\kappa = \pi.\rho$.
    Then, each rewrite step in $t_{0,i}$
    is mirrored  in $s_{0,i}$.
    
    To be precise, let
    $\flat(t_{0,i+1}) = \flat(t_{0,i}[r' \sigma']_{\tau})$ using a rule $\ell' \to r' \in \flat(\C{P} \cup
    \C{P}^{=})$, the substitution $\sigma'$, and the position $\tau$.
    If $\tau$ is above or parallel to the position $\kappa$ of the
    subterm that corresponds to our
    classical chain, then we do nothing and set $s_{0,i} =
    s_{0,i+1}$.
    However, a rewrite step on or above $\kappa$ might change the position of this subterm,
    i.e., the position that we denoted with $\kappa$ may be modified in each such step. Note
    however, that this subterm cannot be erased by such a step because then we would remove
    the only annotated symbol and the chain would become finite.
    If $\tau$ is below $\kappa$ (i.e., we have $\tau = \kappa.\pi'$ for some position $\pi'$), then we have $\flat(s_{0,i}) \trianglelefteq_{\#}^{\kappa} t_{0,i}$, and can rewrite $s_{0,i}$
    with the rule $\ell' \to \flat(r')$, the substitution $\sigma'$, and the position $\pi'$.
    If $\kappa$ denotes the current (possibly changed) position of the corresponding subterm,
    then
    we get $\flat(s_{0,i+1}) = \flat(s_{0,i}[r' \sigma']_{\pi'}) \trianglelefteq_{\#}^{\kappa}
    t_{0,i+1}$.
    Finally, $\tau$ cannot be $\kappa$, since then this would not be a $(\mathbf{r})$-rewrite step but a $(\mathbf{pr})$-rewrite step.
    In the end, we obtain $\flat(s_1) \trianglelefteq_{\#} t_1$.

    We can now repeat this for each $i \in \IN$ and result in our desired chain.
\end{myproof}

\DerelProcTwo*

\begin{myproof}
    \underline{Soundness}:
    By \Cref{def:relative-rewrite-chain}, $(\C{P},
    \C{P}^{=})$ is not SN iff there exists an infinite $(\C{P}, \C{P}^{=})$-chain. Such a
    chain would also be an infinite $(\C{P}\cup \C{P}^{=}_a, \C{P}^{=}_b)$-chain.
    There is an infinite $(\C{P}\cup \C{P}^{=}_a, \C{P}^{=}_b)$-chain iff there is an infinite
    $(\C{P}\cup \mathtt{split}(\C{P}^{=}_a), \C{P}^{=}_b)$-chain as we only need at most a single annotation in the main ADPs. 
    By \Cref{def:relative-rewrite-chain}, this is
    equivalent to non-termination of $(\C{P}\cup \mathtt{split}(\C{P}^{=}_a), \C{P}^{=}_b)$.
\end{myproof}

Before we prove the soundness and completeness of the dependency graph processor, we
present
another definition regarding origin graphs.
In \Cref{def:orig} we 
have seen that for a \emph{non-annotated} rewrite sequence $t_0 \to_{\R \cup \R^{=}} \ldots$ one
can obtain
several different origin graphs.
Now, we define the \emph{canonical} origin graph for an \emph{annotated} rewrite sequence
$t_0 \tored{}{}{\C{P} \cup \C{P}^{=}} \ldots$\
This graph 
represents the flow of the annotations in this sequence.

\begin{definition}[Canonical Origin Graph]\label{def:can_orig}
    Let $(\C{P}, \C{P}^=)$ be an ADP problem
    and let $\Theta: t_0 \tored{}{}{\C{P} \cup \C{P}^{=}} t_1 \tored{}{}{\C{P} \cup
      \C{P}^{=}} \ldots$\
    The \defemph{canonical origin graph} for $\Theta$
    has the nodes $(i,\pi)$  for all $i \in \IN$ and all $\pi \in \pos(t_i)$,
    and its edges are defined as follows:
  For $i \in \IN$,
    let
    the step  $t_i \tored{}{}{\C{P} \cup \C{P}^{=}} t_{i+1}$
 be performed using the rule
   $\ell \to r \in \C{P} \cup \C{P}^{=}$, the position $\tau$, the substitution $\sigma$, and the VRF $\varphi$.
    Furthermore, let $\pi \in \pos(t_i)$.
    \begin{itemize}
        \item[(a)] If $\pi < \tau$ or $\pi \bot \tau$ (i.e., $\pi$ is above or parallel to $\tau$),        
        then there is an edge from $(i,\pi)$ to $(i+1,\pi)$.
        \item[(b)] For $\pi=\tau$,
        there is an edge from $(i,\pi)$ to $(i+1,\pi.\alpha)$ for all $\alpha \in \posT(r)$.
        \item[(c)] If $\pi=\tau.\alpha_\ell.\beta$
        for a variable position $\alpha_\ell \in \pos_\VSet(\ell)$ and $\beta \in \NN^*$, then
        there is an edge from $(i,\tau.\alpha_\ell.\beta)$ to 
        $(i+1,\tau.\varphi(\alpha_\ell).\beta)$ if $\varphi(\alpha_\ell) \neq \bot$.
        \item[(d)] For all other positions $\pi \in \pos(t_i)$, there is no outgoing edge from the node $(i, \pi)$.
    \end{itemize}
    Moreover, if an ADP is applied with Case $(\mathbf{pr})$ at position $\tau$ in
   $t_i$ in   $\Theta$, then all edges originating in $(i, \tau)$ are labeled with this ADP. All other
    edges are not labeled.
\end{definition}

    So for $\R_2$ from \Cref{example:redex-creating}
    where $\ADPairMain{\R_2}$ consists of
    \begin{equation}\label{mainR2} 
        \ta \to \tb
    \end{equation}
    and $\ADPairBase{\R_2^{=}}$ consists of
    \begin{equation}\label{baseR2}
        \tf \to \tc(\tF,\tA),
    \end{equation}
    the chain from \Cref{ex:ADPs-for-redex-creation-1} yields the following canonical origin graph:
    \[ \xymatrix @-1.5pc @C=1pc{
    t_0=&   & &   & &\tF\phantom{,}\ar@{-}[dllll]_{\eqref{baseR2}}\ar@{-}[d]^{\eqref{baseR2}}\ar@{-}[drr]^{\eqref{baseR2}}     & &  \\
    t_1=&\td\ar@{-}[d]&(&   & &\tF,\ar@{-}[d] &    &\tA\ar@{-}[d]^{\eqref{mainR2}})\\
    t_2= &\td\ar@{-}[d]&(&   & &\tF\ar@{-}[dll]_{\eqref{baseR2}}\ar@{-}[d]_{\eqref{baseR2}}\ar@{-}[dr]^{\eqref{baseR2}},    & &\tb\ar@{-}[d])\\
    t_3=&\td\ar@{-}[d]&(&\td\ar@{-}[d]&(&\tF,\ar@{-}[d],&\tA\ar@{-}[d]^{\eqref{mainR2}})&\tb\ar@{-}[d])\\
    t_4=&\td&(&\td&(&\tF,&\tb),&\tb)
    }
    \]

    For $\R_3$ from \Cref{example:redex-creatingAbove}
    where $\ADPairMain{\R_3}$ consists of
    \begin{equation}\label{mainR3} 
        \ta(x) \to \tb(x)
    \end{equation}
    and $\ADPairBase{\R_2^{=}}$ consists of
    \begin{equation}\label{baseR3}
        \tf \to \tA(\tF),
    \end{equation}
    the chain from \Cref{ex:ADPs-for-redex-creation-2} yields the following canonical origin graph:
    \[ \xymatrix @-1.5pc @C=1pc{
    t_0= &  &  &  & & \tF\ar@{-}[dllll]_{\eqref{baseR3}}\ar@{-}[d]^{\eqref{baseR3}} &  \\
    t_1= & \tA\ar@{-}[d]_{\eqref{mainR3}} & ( &  & & \tF\ar@{-}[d] & ) \\
    t_2= & \tb\ar@{-}[d] & ( &  & & \tF\ar@{-}[dll]_{\eqref{baseR3}}\ar@{-}[d]^{\eqref{baseR3}} & ) \\
    t_3= & \tb\ar@{-}[d] & ( & \tA\ar@{-}[d]_{\eqref{mainR3}} & ( & \tF\ar@{-}[d] & )) \\
    t_4= & \tb & ( & \tb & ( & \tF & ))
    }
    \]


\RelativeDepGraphProc*

\begin{myproof}
    \underline{Completeness}:
    For every $(\C{P}', {\C{P}^{=}}')$-chain with $(\C{P}', {\C{P}^{=}}') \in
    \Proc_{\mathtt{DG}}(\C{P},\C{P}^{=})$ there exists a $(\C{P}, \C{P}^{=})$-chain with
    the same terms and possibly more annotations.
    Hence, if some ADP problem in $\Proc_{\mathtt{DG}}(\C{P},\C{P}^{=})$ is not SN, then
    $(\C{P}, \C{P}^{=})$ is
    not SN either.

    \smallskip

    \noindent
    \underline{Soundness}:
    By the definition of $\tored{}{}{}$, whenever there is a path from an edge labeled with
    an ADP $\ell \to r$ to an edge labeled with an ADP 
    $\ell' \to r'$ in the canonical origin graph, then there is a path from 
    $\ell \to r$ to
    $\ell' \to r'$ in the dependency graph. Since there only exist finitely many ADPs and the
    chain uses ADPs from $\PP$ with Case $(\mathbf{pr})$ infinitely many times, there are
    two cases:

    Either there exists a path in the canonical origin graph where infinitely many edges are labeled
    with an ADP from $\PP$. Then after finitely many steps, this path only uses edges
    labeled with ADPs from an SCC $\QQ$ of the dependency graph that contains an ADP from $\PP$
    (i.e., from a $\QQ
    \in \mathtt{SCC}_{\C{P}}^{(\C{P}, \C{P}^{=})}$).

    Otherwise, there is no such path in the canonical origin graph, but then there is a path in the
    canonical origin graph where infinitely many edges are labeled with ADPs from $\PP^=$ and this path
    generates infinitely many paths that lead to an edge labeled with an ADP from $\PP$. 
    Then after finitely many steps,
    this path only uses edges
    labeled with ADPs from a minimal lasso of the dependency graph (i.e., from a
    $\QQ \in \mathtt{Lasso}$).

    So in both cases, there exists a $\QQ \in \mathtt{SCC}_{\C{P}}^{(\C{P}, \C{P}^{=})} \cup \mathtt{Lasso}$
    such that the infinite path gives rise to an infinite $(\,(\C{P} \cap \QQ) \cup \flat(\C{P} \setminus \QQ), (\C{P}^= \cap \QQ) \cup \flat(\C{P}^{=} \setminus \QQ \,)\,)$-chain.
    Since the ADPs $\flat(\PP \setminus \QQ)$ are never used for steps with 
    Case $(\mathbf{pr})$ in this infinite chain, they can also be moved to the base ADPs. Thus, this is also an
    infinite $(\,\C{P} \cap \QQ, (\C{P}^= \cap \QQ)\cup \flat( \,(\C{P} \cup \C{P}^{=}) \setminus \QQ \,)\,)$-chain, 
    i.e., $(\,\C{P} \cap \QQ, (\C{P}^= \cap \QQ)\cup \flat( \,(\C{P} \cup \C{P}^{=}) \setminus \QQ \,)\,)$ is not SN either.
\end{myproof}

\RelRPP*

\begin{myproof}
    \underline{Completeness}:
    For every $(\C{P} \setminus \PP_{\succ}, (\C{P}^{=} \setminus \PP_{\succ}) \cup
        \flat(\PP_{\succ}))$-chain there exists a $(\C{P}, \C{P}^{=})$-chain with the same
        terms and possibly
        more annotations.
        Hence, if
$(\C{P} \setminus \PP_{\succ}, (\C{P}^{=} \setminus \PP_{\succ}) \cup
        \flat(\PP_{\succ}))$
 is not SN, then $(\C{P}, \C{P}^{=})$ is not SN either.

    \smallskip

    \noindent
    \underline{Soundness}:
    We start by showing that the conditions of the theorem extend to rewrite steps instead of just ADPs:
    \begin{enumerate}
        \item[(a)] If $s,t \in \TSet{\Sigma^\#}{\VSet}$ with $s \tored{}{}{\C{P} \cup \C{P}^{=}} t$, then $\subA(s) \succsim \subA(t)$.
        \item[(b)] If $s,t \in \TSet{\Sigma^\#}{\VSet}$ with $s \tored{}{}{\PP_{\succ}} t$ using Case $(\mathbf{pr})$, then $\subA(s) \succ \subA(t)$.
    \end{enumerate}
    For this, we extend $\subA(t)$ to terms with possibly more than two annotations by defining
    $\subA(t) = \Com{2}(r_1^\#,\ldots,\Com{2}(r_{n-1}^\#,r_n^\#)\ldots)$ 
    if $r_i \trianglelefteq_{\#}^{\pi_i} r$ for $n \geq 2$ and the
    positions $\pi_1, \ldots, \pi_n$ with $\pi_i <_{lex} \pi_{i+1}$ for all $1 \leq i < n$.
   

    \begin{itemize}
        \item[(a)] Assume that we have $s \tored{}{}{\C{P} \cup \C{P}^{=}} t$ using the
        ADP $\ell \to r$, the VRF $\varphi$, the position $\pi$, and the substitution
        $\sigma$. So we have $\flat(s|_{\pi}) = \ell \sigma$.
        Furthermore, let $\subA(s)$ contain the terms $s_1^\#, \ldots, s_n^\#$
        with annotated root symbols. If $n=0$, then we have $\subA(s) = \Com{0} = \subA(t)$ which
        proves the claim.

        Otherwise, 
        we partition $s_1,\ldots,s_n$ into several disjoint groups:

        Let $s_{i_1}, \ldots, s_{i_{n_1}}$ be all those $s_{i}$ that are at
        positions above $\pi$ in $s$ (i.e., these are those $s_i$ where  $\pi_{i} < \pi$).
        Let $s_{i_{n_1 + 1}}, \ldots, s_{i_{n_2}}$ be all those $s_{i}$
        that are at positions parallel to $\pi$ or on or below a variable position of $\ell$ that
        is ``considered'' by the VRF (i.e., those $s_i$ where
        $\pi_{i} \bot \pi$ or $\pi.\alpha \leq \pi_{i}$ for some $\alpha \in \pos_{\VSet}(\ell)$ with $\varphi(\alpha) \neq \bot$).
        Finally, let $s_{i_{n_2 + 1}}, \ldots, s_{i_{n_3}}$ be all those $s_{i}$
        that are below position $\pi$, but not on or below a variable position of $\ell$ that
        is ``considered'' by the VRF (i.e., those $s_i$ where
        $\pi_i = \pi.\alpha$ for some $\alpha \in \pos_{\Sigma}(\ell)$ with $\alpha \neq \epsilon$
        or $\pi.\alpha \leq \pi_{i}$ for some $\alpha \in \pos_{\VSet}(\ell)$ with $\varphi(\alpha) = \bot$).

        If the rewrite step takes place at a position that is not annotated (i.e., the symbol at
        position $\pi$ in $s$ is not annotated), then we have $n_3 =
        n$ and $\{ i_1, \ldots, i_{n_3}\} = \{1,\ldots, n\}$. 
        Otherwise, we have $n_3 = n-1$ and $\{s_{i_1},
        \ldots, s_{i_{n_3}}, \ell^\#\sigma\} = \{s_1, \ldots, s_n \}$.

        After the rewrite step with $\tored{}{}{\C{P} \cup \C{P}^{=}}$,
        $\subA(t)$
        contains the following terms with annotated root symbols:
        The terms
        $s_{i_{n_1 + 1}}, \ldots, s_{i_{n_2}}$ are unchanged and still contained in 
        $\subA(t)$. The terms $s_{i_{n_2 + 1}}, \ldots, s_{i_{n_3}}$ are removed, i.e., they are
        no longer in $\subA(t)$. 
        The terms $s_{i_1}, \ldots, s_{i_{n_1}}$ are replaced by 
        $s_{i_1}[\flat(r) \sigma]_{\tau_{1}}, \ldots, s_{i_{n_1}}[\flat(r) \sigma]_{\tau_{n_1}}$
        for appropriate positions $\tau_i \neq \varepsilon$.
        Furthermore, if
        the symbol at position $\pi$ in $s$ was annotated, then in addition, $\subA(t)$ contains
        the terms $r_1^\#\sigma, \ldots, r_m^\#\sigma$ where $r_j$ for $1 \leq j
        \leq m$ are all terms with $r_j \trianglelefteq_{\#} r$.
        Hence, if the symbol at position $\pi$ in $s$ was annotated, then there exist contexts $C, C', C''$
        containing no function symbol except $\Com{2}$ such that


\vspace*{-.2cm}
        
        {\small\[
            \begin{array}{lcl}
              \subA(s) & = & C[s_{i_1}, \ldots, s_{i_{n_1}},s_{i_{n_1 + 1}}, \ldots,
                s_{i_{n_2}},s_{i_{n_2 + 1}}, \ldots, s_{i_{n_3}},\ell^\#\sigma]\\
              & \succsim & C'[s_{i_1}, \ldots, s_{i_{n_1}},s_{i_{n_1 + 1}}, \ldots,
                s_{i_{n_2}},s_{i_{n_2 + 1}}, \ldots, s_{i_{n_3}},r_1^\#\sigma, \ldots,
                r_m^\#\sigma]\\
              &&\qquad\qquad \text{by $\ell^\# \succsim \subA(r)$, $\Com{}$-invariance, and $\Com{}$-monotonicity}\\
              & \succsim & C''[s_{i_1}, \ldots, s_{i_{n_1}},s_{i_{n_1 + 1}}, \ldots,
                s_{i_{n_2}},r_1^\#\sigma, \ldots,
                r_m^\#\sigma]\\
               & \succsim & C''[s_{i_1}[\flat(r) \sigma]_{\tau_{1}}, \ldots, s_{i_{n_1}}[\flat(r) \sigma]_{\tau_{n_1}},s_{i_{n_1 + 1}}, \ldots,
                s_{i_{n_2}},r_1^\#\sigma, \ldots,
                r_m^\#\sigma]\\
              &&\qquad\qquad \text{by $\ell \succsim \flat(r)$,
                $\Com{}$-invariance, and $\Com{}$-monotonicity}\\
                            & = & \subA(t)\!    
            \end{array}
            \]}


\vspace*{-.2cm}

        
        \noindent
        If the symbol at position $\pi$ in $s$ was not annotated, then we obtain


\vspace*{-.2cm}

        
        {\small\[
          \begin{array}{lcl}
    \subA(s) & = & C[s_{i_1}, \ldots, s_{i_{n_1}},s_{i_{n_1 + 1}}, \ldots,
                s_{i_{n_2}},s_{i_{n_2 + 1}}, \ldots, s_{i_{n_3}}]\\
                & \succsim & C'[s_{i_1}, \ldots, s_{i_{n_1}},s_{i_{n_1 + 1}}, \ldots,
                s_{i_{n_2}}]\\
               & \succsim & C'[s_{i_1}[\flat(r) \sigma]_{\tau_{1}}, \ldots, s_{i_{n_1}}[\flat(r) \sigma]_{\tau_{n_1}},s_{i_{n_1 + 1}}, \ldots,
                s_{i_{n_2}}]\\
              &&\qquad\qquad \text{by $\ell \succsim \flat(r)$, $\Com{}$-invariance, and $\Com{}$-monotonicity}\\
                            & = & \subA(t)\!    
            \end{array}
          \]}


        
        \item[(b)] Assume that we have $s \tored{}{}{\PP_{\succ}} t$ using the ADP $\ell \to r$, the VRF $\varphi$, the position $\pi$, the substitution $\sigma$, and we rewrite at an annotated position.
          So here, $\subA(s) \neq \Com{0}$.   Using the same notations as in (a), we get
          

\vspace*{-.2cm}

          {\small\[
            \begin{array}{lcl}
              \subA(s) & = & C[s_{i_1}, \ldots, s_{i_{n_1}},s_{i_{n_1 + 1}}, \ldots,
                s_{i_{n_2}},s_{i_{n_2 + 1}}, \ldots, s_{i_{n_3}},\ell^\#\sigma]\\
              & \succ & C'[s_{i_1}, \ldots, s_{i_{n_1}},s_{i_{n_1 + 1}}, \ldots,
                s_{i_{n_2}},s_{i_{n_2 + 1}}, \ldots, s_{i_{n_3}},r_1^\#\sigma, \ldots,
                r_m^\#\sigma]\\
              &&\qquad\qquad \text{by $\ell^\# \succ \subA(r)$, $\Com{}$-invariance, and $\Com{}$-monotonicity}\\
              & \succsim & C''[s_{i_1}, \ldots, s_{i_{n_1}},s_{i_{n_1 + 1}}, \ldots,
                s_{i_{n_2}},r_1^\#\sigma, \ldots,
                r_m^\#\sigma]\\
               & \succsim & C''[s_{i_1}[\flat(r) \sigma]_{\tau_{1}}, \ldots, s_{i_{n_1}}[\flat(r) \sigma]_{\tau_{n_1}},s_{i_{n_1 + 1}}, \ldots,
                s_{i_{n_2}},r_1^\#\sigma, \ldots,
                r_m^\#\sigma]\\
              &&\qquad\qquad \text{by $\ell \succsim \flat(r)$, $\Com{}$-invariance, and
                $\Com{}$-monotonicity}\\
                            & = & \subA(t)\!    
            \end{array}
        \]}
    \end{itemize}

    We can now prove soundness.
    Assume that $(\C{P}, \C{P}^{=})$ is not SN.
    Then there exists an infinite $(\C{P}, \C{P}^{=})$-chain $t_0 \tored{}{}{\C{P} \cup \C{P}^{=}} t_1 \tored{}{}{\C{P} \cup \C{P}^{=}} t_2 \ldots$
    If the chain uses an infinite number of rewrite steps with rules from $\PP_{\succ}$ and Case $(\mathbf{pr})$, 
    then $\subA(t_0) \succsim \subA(t_1) \succsim \subA(t_2) \succsim \ldots$
    would contain an infinite number of steps where the
    strict relation $\succ$ holds, which is a contradiction to well-foundedness of $\succ$, 
    as $\succ$ is compatible with $\succsim$.

    Hence, the chain only contains a finite number of $\tored{}{}{\PP_{\succ}}$-steps with
    Case $(\mathbf{pr})$. So there is an infinite suffix of the chain where only
    ADPs from $(\C{P} \setminus \PP_{\succ}) \cup (\C{P}^{=} \setminus \PP_{\succ})$ are
    used with
    Case $(\mathbf{pr})$.
    This means that $t_i \tored{}{}{\C{P} \cup \C{P}^{=}} t_{i+1} \tored{}{}{\C{P} \cup
      \C{P}^{=}} t_{i+2} \ldots$ is an infinite $((\C{P} \setminus \PP_{\succ}) \cup
        \flat(\PP \cap \PP_{\succ}), (\C{P}^{=} \setminus \PP_{\succ}) \cup
        \flat(\C{P}^{=} \cap \PP_{\succ}))$-chain, as ADPs that are only used for steps
        with Case $(\mathbf{r})$ do not need annotations.
Since the ADPs  $\flat(\PP \cap \PP_{\succ})$ are never used for steps with 
Case $(\mathbf{pr})$, they can also be moved to the base ADPs. Thus, this is also an
infinite $(\C{P} \setminus \PP_{\succ}, (\C{P}^{=} \setminus \PP_{\succ}) \cup
        \flat(\PP_{\succ}))$-chain, i.e., 
$(\C{P} \setminus \PP_{\succ}, (\C{P}^{=} \setminus \PP_{\succ}) \cup
        \flat(\PP_{\succ}))$ is not SN either.
\end{myproof}


Since ADPs are ordinary rewrite rules with annotations, we can also use
ordinary reduction orderings (that are closed under contexts)
to remove rules from an ADP problem completely.

\begin{restatable}[Rule Removal Processor]{theorem}{RelRRP}\label{thm:RelRRP}
    Let $(\C{P},\C{P}^{=})$ be an ADP problem, and let
    $(\succsim, \succ)$ be a
 reduction pair where $\succ$ is closed under contexts
    such that $\flat(\C{P} \cup \C{P}^{=}) \subseteq {\succsim}$.
  Moreover, let $\PP_{\succ} \subseteq \C{P} \cup \C{P}^{=}$ 
  such that $\flat(\C{P}_{\succ}) \subseteq {\succ}$.
Then
    $\Proc_{\mathtt{RR}}(\C{P},\C{P}^{=}) = \{(\C{P} \setminus \PP_{\succ}, \C{P}^{=} \setminus \PP_{\succ})\}$
 is sound and complete.
\end{restatable}

\begin{myproof}
    \underline{Completeness}:
    Every $(\C{P} \setminus \PP_{\succ}, \C{P}^{=} \setminus \PP_{\succ})$-chain 
    is also a $(\C{P},\C{P}^{=})$-chain.
    Hence, if $(\C{P} \setminus \PP_{\succ}, \C{P}^{=} \setminus \PP_{\succ})$ is not SN, then
    $(\C{P},\C{P}^{=})$ is not SN either.

    \smallskip

    \noindent
    \underline{Soundness}:
    Assume that $(\C{P},\C{P}^{=})$ is not SN.
    Let $t_0, t_1, \ldots$ be an infinite $(\C{P},\C{P}^{=})$-chain.
    Then, $t_i \tored{}{}{\C{P} \cup \C{P}^{=}} t_{i+1}$ for all $i \in \IN$.
    Since $\flat(\C{P} \cup \C{P}^{=}) \subseteq  {\succsim}$ and
    $\succsim$ is closed under contexts and substitutions, we obtain
     $\flat(t_0) \succsim \flat(t_1) \succsim \ldots$
    Assume for a contradiction that we use infinitely many steps with
$\tored{}{}{\C{P}_{\succ}}$.
    Then, $\flat(t_0) \succsim \flat(t_1) \succsim \ldots$
 would contain an infinite number of steps where the
 strict relation $\succ$ holds, because 
$\flat(\C{P}_{\succ}) \subseteq {\succ}$ and $\succ$  is also closed under contexts and
 substitutions. This
is a contradiction to well-foundedness of $\succ$, 
    as $\succ$ is compatible with $\succsim$.

  Hence, the chain only contains a finite number of $\tored{}{}{\PP_{\succ}}$-steps. So there is an infinite suffix of the chain where only
  ADPs from
$(\C{P} \cup \C{P}^{=})\setminus \PP_{\succ}$ are
  used.
      This means that $t_i \tored{}{}{\C{P} \cup \C{P}^{=}} t_{i+1} \tored{}{}{\C{P} \cup
        \C{P}^{=}} t_{i+2} \ldots$ is an infinite
      $(\C{P} \setminus \PP_{\succ}, \C{P}^{=} \setminus \PP_{\succ})$-chain.
Hence,
$(\C{P} \setminus \PP_{\succ}, \C{P}^{=} \setminus \PP_{\succ})$ is not SN either.
\end{myproof}
