\section{Relative Term Rewriting}\label{Relative Term Rewriting}

We assume familiarity with term rewriting \cite{baader_nipkow_1999} and regard (finite) TRSs
over a (finite) signature $\Sigma$ and a set of variables $\VSet$.

\begin{example}\label{ex:divlTRS}
Consider the following TRS $\R_{\tdivl}$, where $\tdivl(x,\xs)$ computes the
number that results from dividing $x$ by each element of the list $\xs$. 
As usual, natural numbers are 
represented by the function symbols $\O$ and $\ts$, and lists are represented via $\tnil$
and $\tcons$. 
Then $\tdivl(\ts^{24}(\O), \tcons(\ts^4(\O), \tcons(\ts^3(\O),\tnil)))$ evaluates to
$\ts^2(\O)$, because $(24/4)/3 = 2$. Here, $\ts^2(\O)$ stands for $\ts(\ts(\O))$, etc.

\vspace*{-0.5cm}

{\footnotesize
\hspace*{-0.85cm}
\begin{minipage}[t]{5.1cm}
    \begin{align}
        \label{R-minus-rule-2} \tminus(x,\O) &\to x\\
        \label{R-minus-rule-1} \tminus(\ts(x),\ts(y)) &\to \tminus(x,y) \\
        \label{R-div-rule-1} \tdiv(x,\ts(\O)) &\to x\!
    \end{align}
\end{minipage}\hspace{.3cm}
\begin{minipage}[t]{7.1cm}
    \begin{align}
        \label{R-div-rule-2} \tdiv(\ts(x),\ts(y)) &\to \ts(\tdiv(\tminus(x,y),\ts(y)))\\
        \label{R-list-rule-2} \tdivl(x,\tnil) &\to x \\
        \label{R-list-rule-1} \tdivl(x,\tcons(y,\xs)) &\to \tdivl(\tdiv(x,y),\xs) \!
         \end{align}
\end{minipage}}
\end{example}

\smallskip

\noindent 
A TRS $\R$ induces a \emph{rewrite relation} ${\to_{\R}} \subseteq \TSet{\Sigma}{\VSet}
\times \TSet{\Sigma}{\VSet}$ on terms where $s \to_{\R} t$ holds if there is a position
 $\pi \in \pos(s)$, 
a rule $\ell \to r \in \R$, and a substitution $\sigma$ such that $s|_{\pi}=\ell\sigma$ and $t = s[r\sigma]_{\pi}$.
For example, we have $\tminus(\ts(\O),\ts(\O)) \to_{\R_{\tdivl}} \tminus(\O,\O) \to_{\R_{\tdivl}} \O$.
We call a TRS $\R$ \emph{strongly normalizing (SN)} or \emph{terminating} if $\to_{\R}$ is
well founded. Using the DP framework, one can easily prove that
$\R_{\tdivl}$ is SN (see \Cref{Dependency Pairs for Ordinary Term Rewriting}). In
particular, in each application of the recursive $\tdivl$-rule 
\eqref{R-list-rule-1}, the length of the list in $\tdivl$'s second argument is
decreased by one.

In the relative setting, one considers two TRSs
$\R$ and $\R^{=}$.
We say that $\R$ is \emph{relatively strongly normalizing} w.r.t.\ $\R^{=}$ (i.e.,
$\R / \R^{=}$ is SN) if there is no infinite $(\to_{\R} \cup \to_{\R^{=}})$-rewrite sequence that uses an infinite number of $\to_{\R}$-steps.
We refer to $\R$ as the \emph{main} and $\R^{=}$ as the \emph{base} TRS.

\begin{example}\label{ex:mset1}
    For example, let $\R_{\tdivl}$ be the \emph{main} TRS.
    Since the order of the list elements does not affect \pagebreak[2] the termination
   of $\R_{\tdivl}$, 
     this algorithm also works for multisets.
   To abstract lists to multisets, we add the \emph{base} TRS $\R^{=}_{\tset} = \{\eqref{B-com-rule-1}\}$.
    \begin{equation}
        \label{B-com-rule-1} \tcons(x, \tcons(y,\zs)) \to \tcons(y, \tcons(x,\zs)) 
    \end{equation}
    $\R^{=}_{\tset}$ is non-terminating, since it can
    switch elements in a list arbitrarily often.
    How-\linebreak ever, $\R_{\tdivl} / \R^{=}_{\tset}$ is SN as each application of
    Rule \eqref{R-list-rule-1} still reduces the list length.
\end{example}

We will use the following four examples to show why a naive
adaption of dependency pairs does not work in the relative setting and why we need our new
notion of \emph{annotated dependency pairs}.
These examples represent different types of infinite rewrite sequences
that lead to non-termination in the relative setting: \emph{redex-duplicating},
\emph{redex-creating} (or ``-emitting''), and \emph{ordinary infinite sequences}.

\begin{example}[Redex-Duplicating]\label{example:redex-duplicating}
    Consider the TRSs $\R_1 = \{\ta \to \tb\}$ and $\R_1^{=} = \{\tf(x) \to \tc(\tf(x),x)\}$.
    $\R_1 / \R_1^{=}$ is not SN due to the infinite rewrite sequence $\underline{\tf(\ta)}
    \to_{\R_1^{=}}\linebreak \tc(\tf(\ta),\underline{\ta}) \to_{\R_1} \tc(\underline{\tf(\ta)},\tb) \to_{\R_1^{=}}
    \tc(\tc(\tf(\ta),\underline{\ta}),\tb)\to_{\R_1}
      \tc(\tc(\tf(\ta),\tb),\tb)\to_{\R_1^{=}}
    \ldots$\
    The reason is that $\R_1^{=}$ can be used to duplicate an arbitrary $\R_1$-redex infinitely often.
\end{example}

\begin{example}[Redex-Creating on Parallel Position]\label{example:redex-creating}
    Next, consider $\R_2 = \{\ta \to \tb\}$ and $\R_2^{=} = \{\tf \to \tc(\tf,\ta)\}$.
    $\R_2 / \R_2^{=}$ is not SN as we have the infinite rewrite sequence $\underline{\tf}
    \to_{\R_2^{=}} \tc(\tf,\underline{\ta}) \to_{\R_2} \tc(\underline{\tf},\tb) \to_{\R_2^{=}}
    \tc(\tc(\tf,\underline{\ta}),\tb) \to_{\R_2}
 \tc(\tc(\underline{\tf},\tb),\tb) \to_{\R_2^{=}} \ldots$\
 Here, $\R_2^{=}$ can create an
 $\R_2$-redex infinitely often (where
 in the right-hand side $\tc(\tf,\ta)$ of $\R_2^{=}$'s rule, the
 $\R_2^=$-redex $\tf$ and
 the created $\R_2$-redex $\ta$ are on parallel positions).
\end{example}

\begin{example}[Redex-Creating on Position Above]\label{example:redex-creatingAbove}
  Let $\R_3 = \{\ta(x) \to \tb(x)\}$ and $\R_3^{=} = \{\tf \to \ta(\tf)\}$.
    $\R_3 / \R_3^{=}$ is not SN as we have $\underline{\tf} \to_{\R_3^{=}}
    \underline{\ta}(\tf) \to_{\R_3} \tb(\underline{\tf}) \to_{\R_3^{=}}
    \tb(\underline{\ta}(\tf)) \to_{\R_3} 
 \tb(\tb(\underline{\tf})) \to_{\R_3^{=}}\ldots$, i.e., again
    $\R_3^{=}$ can be used to create an
 $\R_3$-redex infinitely often.
 In the right-hand side $\ta(\tf)$ of
$\R_3^=$'s rule,
the position  of the created $\R_3$-redex $\ta(\ldots)$
    is above the position of the  $\R_3^=$-redex $\tf$.
\end{example}

  

\begin{example}[Ordinary Infinite]\label{example:ordinary-infinite}
  Finally, consider $\R_4 = \{\ta \to \tb\}$ and $\R_4^{=} = \{ \tb \to \ta\}$.
Here, the base TRS $\R_4^{=}$ can neither duplicate nor create an $\R_4$-redex infinitely often,
    but in combination with the main TRS $\R_4$ we obtain the
  infinite rewrite sequence $\ta \to_{\R_4}
    \tb \to_{\R_4^{=}} \ta \to_{\R_4}
    \tb \to_{\R_4^{=}} \ldots$\ Thus,
    $\R_4 / \R_4^{=}$ is not SN. 
\end{example}

