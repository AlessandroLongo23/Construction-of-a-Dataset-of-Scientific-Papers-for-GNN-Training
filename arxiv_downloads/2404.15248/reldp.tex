\section{Annotated Dependency Pairs for Relative Termination}\label{Relative DP Framework}



As shown in \Cref{Dependency Pairs for Relative Termination},
up to now there only exist criteria \cite{iborra2017relative} that state
when it is sound to apply
\emph{ordinary} DPs for proving relative termination, but there is no \emph{specific} DP-based technique
to analyze relative termination directly. To solve this problem, we now
adapt the concept of \emph{annotated dependency pairs} (ADPs) for relative termination.
ADPs were introduced in \cite{FLOPS2024} to prove innermost almost-sure termination of probabilistic term rewriting.
In the relative setting, we can use similar dependency pairs as in the probabilistic setting,
but with a different rewrite relation $\tored{}{}{}$ to deal with non-innermost rewrite steps.
Compared to \cite{iborra2017relative},
we can (a) remove the requirement of dominance, which will be handled
by the dependency graph processor, and (b) allow for ADP processors that are specifically designed for the relative
setting before possibly moving to ordinary DPs.
The requirement that 
$\R^{=}$ must be non-duplicating remains, since DPs do not help in analyzing
redex-duplicating sequences as in \Cref{example:redex-duplicating}, where the crucial redex
$\ta$ is not generated from a ``function call'' \pagebreak[2]
in the right-hand side of a rule,
but it just corresponds to a duplicated variable.
To handle TRSs $\R / \R^{=}$ where $\R^{=}_{dup} \subseteq\R^{=}$
is duplicating,
one can move the duplicating rules to the main TRS $\R$ and 
try to prove relative termination of
$(\R \cup \R^{=}_{dup})/(\R^{=} \setminus \R^{=}_{dup})$ instead,
or one can try to find a reduction pair $(\succsim, \succ)$ where
$\succ$ is closed under contexts such that $\R \cup \R^= \subseteq {\succsim}$ and
$\R^{=}_{dup} \subseteq {\succ}$. Then it suffices to prove relative termination of $(\R
\setminus \succ, \R^{=} \setminus \succ)$ instead.



For ordinary termination, we create a separate DP for each occurrence of a
defined symbol in the right-hand side of a rule (and no DP is created for rules without
defined symbols in their right-hand sides).
This would work to detect
\emph{ordinary infinite} sequences
like the one in \Cref{example:ordinary-infinite}
in the relative setting, i.e., such an infinite sequence would give rise to an infinite chain. 
However,  as shown in \Cref{ex:dps-dont-work-in-relative}, this
would not suffice to detect infinite redex-creating sequences as in
\Cref{example:redex-creating}
with   $\R_2 = \{\ta \to \tb\}$ and $\R_2^{=} = \{\tf \to \tc(\tf,\ta)\}$:
$\underline{\tf}
    \to_{\R_2^{=}} \tc(\tf,\underline{\ta}) \to_{\R_2} \tc(\underline{\tf},\tb) \to_{\R_2^{=}}
    \tc(\tc(\tf,\underline{\ta}),\tb) \to_{\R_2}
     \ldots$\

  

    
Here, (1) we need a DP for
the rule $\ta \to \tb$ to detect the reduction of the created
$\R_2$-redex $\ta$, although $\tb$ is a constructor.
Moreover,
(2) both defined symbols $\tf$ and $\ta$ in the right-hand side of
$\tf \to \tc(\tf, \ta)$ have to be considered simultaneously:
We need $\tf$ to create an infinite number of $\R_2$-redexes, and we need $\ta$ since it
is the created $\R_2$-redex.
Hence, for rules from the base TRS $\R_2^=$, we have to consider all possible pairs of
defined symbols in their right-hand sides simultaneously.\footnote{For relative
termination, it suffices to
consider \emph{pairs} of defined symbols.
The reason is that to ``track'' a non-terminating reduction, one only has
to
consider a single redex plus possibly another redex of the base
TRS which may later create a redex again.
}
This is not needed for the main TRS $\R_2$, i.e., if the $\tf$-rule were in the
main TRS, then the $\tf$ in the right-hand side could be considered separately from the $\ta$ that it generates.
Therefore, we distinguish between \emph{main} and \emph{base ADPs} (that are
generated from the main and the base TRS, respectively).



As in \cite{FLOPS2024},  we now annotate defined
symbols directly in the original rewrite rule instead of extracting annotated subterms
from its right-hand side. In this way, we may have terms containing several annotated
symbols, which 
allows us to consider pairs of defined symbols in right-hand sides
simultaneously.

\begin{definition}[Annotations]
    For $t \in \TSet{\SignatureADC}{\VSet}$ 
    and $\Sigma' \subseteq \SignatureADC \cup \VSet$, let $\pos_{\Sigma'}(t)$ be the
    set of all positions of $t$
    with symbols or variables from $\Sigma'$.
    For $\Phi \subseteq \posDT(t)$,
    $\anno_\Phi(t)$ is the variant of $t$ where the symbols at positions from $\Phi$
    are annotated and all\linebreak other annotations are removed.
    Thus, $\posT(\anno_\Phi(t)) = \Phi$, and
    $\anno_\emptyset(t)$ removes all an-\linebreak
    notations from $t$, where we often write
    $\flat(t)$ instead of $\anno_\emptyset(t)$.
    Moreover, for a singleton $\{\pi \}$, we often write $\anno_\pi$ instead of $\anno_{\{\pi\}}$.
   We write $t \trianglelefteq_{\#}^\pi s$ if
there is a $\pi \in \posT(s)$ and $t = \flat(s|_\pi)$
(i.e., $t$ results from  a subterm of $s$ with annotated root
symbol by removing its annotations). We also write $\trianglelefteq_{\#}$ instead of $\trianglelefteq_{\#}^\pi$.
\end{definition}

\begin{example}
    If $\tf \in \SignatureD$, then we have $\anno_{1}(\tf(\tf(x))) =
    \anno_{1}(\tF(\tF(x))) = \tf(\tF(x))$ and $\flat(\tF(\tF(x))) = \tf(\tf(x))$. 
    Moreover, we have $\tf(x) \trianglelefteq_{\#}^{1} \tf(\tF(x))$.
\end{example}


While in \cite{FLOPS2024} all defined symbols on the right-hand sides of rules were
annotated, 
we now define our novel variant of \emph{annotated dependency pairs}
for relative rewriting.

\begin{definition}[Annotated Dependency Pair]\label{def:Canonical-ADPs}
\hspace*{-.2cm}    A rule $\ell\!\to\!r$ with $\ell\!\in\!\TSet{\Sigma}{\VSet} \setminus\linebreak
    \VSet$, $r \in \TSet{\SignatureADC}{\VSet}$, and $\VSet(r) \subseteq \VSet(\ell)$ is
    called an
    \defemph{annotated dependency pair (ADP)}.
 
    Let $\SignatureD$ be the defined symbols of $\R \cup \R^=$, and for $n \in
    \NN$, let $\ADPair{n}{\ell \to r}  =  
    \{\ell \to\linebreak \anno_{\Phi}(r) \mid \Phi \subseteq \pos_{\SignatureD}(r), |\Phi| \leq n\}$.
\pagebreak[2]  The  \defemph{canonical main ADPs} for $\R$ are $\ADPairMain{\R} = \bigcup\limits_{\ell \to r \in \R}
  \!\!\!\!  \ADPairMain{\ell\!\to\!r}$ and  the \defemph{canonical base ADPs} for $\R^=\!$
are  $\ADPairBase{\R^{=}}\!= \!\!\!
    \bigcup\limits_{\ell \to r \in \R^=} \!\!\!\! \ADPairBase{\ell\!\to\!r}$.
\end{definition}

So the left-hand side of an ADP is just the left-hand side of the original rule.
The right-hand side  results from the right-hand side of the original rule
by replacing certain defined symbols $f$ with $f^{\#}$.
Whenever
we have two ADPs  $\ell \to \anno_{\Phi'}(r)$,
$\ell \to \anno_{\Phi}(r)$
with  $\Phi' \subset \Phi$, then
we only consider $\ell \to \anno_{\Phi}(r)$ and remove
 $\ell \to \anno_{\Phi'}(r)$.

\begin{example}\label{ADP-Divl}
    The canonical ADPs
    of \Cref{example:redex-creating} are $\ADPairMain{\R_2} = \{ \ta
    \to \tb\}$
    and $\ADPairBase{\R_2^{=}} = \{\tf \to
    \tc(\tF,\tA)\}$ and for
    \Cref{example:redex-creatingAbove} we get $\ADPairMain{\R_3} = \{ \ta(x)
    \to \tb(x)\}$
    and $\ADPairBase{\R_3^{=}} = \{\tf \to
    \tA(\tF)\}$.
    For $\R_{\tdivl}/\R^{=}_{\tset 2}$ from \Cref{ex:divlTRS,ex:mainExample},
    the ADPs $\ADPairMain{\R_{\tdivl}}$ are

\vspace*{-.4cm}
    
    {\footnotesize
    \hspace*{-.7cm}\begin{minipage}[t]{5.1cm}
        \begin{align}
            \label{R-div-adp-2} \tminus(x,\O) &\to x\\
            \label{R-div-adp-1} \tminus(\ts(x),\ts(y)) &\to \tM(x,y) \\
            \label{R-div-adp-3} \tdiv(x,\ts(\O)) &\to x\\
            \label{R-div-adp-4} \tdivl(x,\tnil) &\to x \!
        \end{align}
    \end{minipage}
    \begin{minipage}[t]{7.5cm}
        \begin{align}
            \label{R-div-adp-5} \tdiv(\ts(x),\ts(y)) &\to \ts(\tD(\tminus(x,y),\ts(y)))\\
            \label{R-div-adp-6} \tdiv(\ts(x),\ts(y)) &\to \ts(\tdiv(\tM(x,y),\ts(y)))\\
            \label{R-div-adp-7} \tdivl(x,\tcons(y,\xs)) &\to \tDL(\tdiv(x,y),\xs) \\
            \label{R-div-adp-8} \tdivl(x,\tcons(y,\xs)) &\to \tdivl(\tD(x,y),\xs) \!
        \end{align}
    \end{minipage}}
    
    \vspace*{-.1cm}
   
  
    \begin{align}
      \label{B-com-adp-2}
      \hspace*{-.2cm}
         \text{and $\ADPairBase{\R^{=}_{\tset 2}}$ contains 
      {\small $\tdivl(z, \tcons(x, \tcons(y,\zs))) \to \tDL(z, \tcons(y, \tcons(x,\zs)))$}}
    \end{align}
\end{example}


In \cite{FLOPS2024},
ADPs were only used for innermost rewriting.
We now modify their rewrite relation and define what happens 
with annotations inside the substitutions during a rewrite step.
To simulate redex-creating sequences as in \Cref{example:redex-creatingAbove}
with ADPs (where the position of the created redex $\ta(\ldots)$
is above the position of the creating redex $\tf$),
ADPs should be able to rewrite above annotated arguments
without removing their annotation (we will demonstrate that in
\Cref{ex:ADPs-for-redex-creation-2}).
Thus, for an ADP $\ell \to\linebreak r$ with $\ell|_\pi = x$, we use a 
\emph{variable reposition function (VRF)} to indicate which occur-\linebreak rence of $x$ in $r$ should
keep the annotations if one rewrites an instance of $\ell$ where the subterm at position
$\pi$ is annotated.
So a VRF maps  positions of variables in the left-hand side of a rule to
positions of the same variable in the right-hand side.

\begin{definition}[Variable Reposition Function]\label{def:Var-Repos-Func}
    Let $\ell \to r$ be an ADP.
	A function $\varphi: \pos_{\VSet}(\ell) \to \pos_{\VSet}(r) \cup \{\bot\}$ is called a
    \defemph{variable reposition function (VRF)} for
    $\ell \to r$
    iff
    $\ell|_\pi = r|_{\varphi(\pi)}$ whenever
 $\varphi(\pi) \neq \bot$.
\end{definition}

\begin{example}\label{example:rel-var-repos-function}
    For the ADP $\ta(x) \to \tb(x)$ for $\R_3$ from \Cref{example:redex-creatingAbove},
    if $x$ on position 1 of the left-hand side is instantiated by $\tF$,
    then the VRF $\varphi(1) = 1$ 
    indicates that
    this ADP rewrites $\tA(\tF)$ to $\tb(\tF)$, whereas
    $\varphi(1) = \bot$  means that
    it rewrites $\tA(\tF)$ to $\tb(\tf)$.
\end{example}

With VRFs we can define the rewrite relation for ADPs w.r.t.\ full rewriting.

\begin{definition}[$\tored{}{}{\C{P}}$]\label{def:ADP-Rewriting}
    Let $\C{P}$ be a set of ADPs.
    A term $s \in \TSet{\SignatureADC}{\VSet}$ rewrites to $t$ using $\C{P}$
    (denoted $s \tored{}{}{\C{P}} t$)
if there is a rule $\ell \to r \in \C{P}$, 
    a substitution $\sigma$, a position $\pi \in \posDT(s)$
    such that $\flat(s|_\pi) = \ell\sigma$, a VRF $\varphi$ for $\ell \to r$,
    and\footnote{\label{ADPComparison2}In \cite{FLOPS2024} there
    were two additional cases in the definition of the corresponding rewrite relation. One of
    them was needed for processors that restrict the set of rules applicable for
    $\mathbf{r}$-steps (e.g., based on usable rules), and the other case 
    was needed to
    ensure that the innermost
    evaluation strategy is not affected by the application of ADP processors. This is
    unnecessary here since we consider full rewriting. On the other hand,
     VRFs are new compared to
\cite{FLOPS2024}, since they are not needed for innermost rewriting.}
    \begin{equation*}
        \begin{array}{rll@{\quad}ll@{\qquad}l}
        t &=                  &s[\anno_{\Phi}(r\sigma)]_{\pi}  & 
        \text{if} & \pi \in \posT(s)    & (\mathbf{pr})\\ 
        t &=                  &s[\anno_{\Psi}(r\sigma)]_{\pi}  & 
        \text{if} & \pi \in\pos_\SignatureD(s) & (\mathbf{r})\!
        \end{array}
    \end{equation*}
    Here, $\Psi \!=\! \{\varphi(\rho).\tau \mid \rho \!\in\! \pos_{\VSet}(\ell), \,
    \varphi(\rho) \neq \bot,  \, \rho.\tau \!\in\! \posT(s|_{\pi}) \}$
    and $\Phi = \posT(r) \cup \Psi$.
\end{definition}
So $\Psi$ considers all positions of annotated symbols in $s|_{\pi}$ that
are below
positions $\rho$ of\linebreak variables in $\ell$. If the VRF maps $\rho$ to a variable position
$\rho'$ in
$r$, then the annotations below $\pi.\rho$ in $s$ are kept in the resulting subterm at
position $\pi.\rho'$
after the rewriting.

Rewriting with $\C{P}$ is like ordinary term rewriting, while considering and
modifying
annotations.
Note that we represent all DPs resulting from a rule as well as the original
rule by just one ADP.  
So the ADP $\tdiv(\ts(x),\ts(y)) \to \ts(\tD(\tminus(x,y),\ts(y)))$
represents both the DP resulting from $\tdiv$ in the right-hand side
of the rule \eqref{R-div-rule-2}, and the rule \eqref{R-div-rule-2} itself 
(by simply disregarding all annotations of the ADP).

Similar to the classical DP framework, our goal is to track specific reduction
sequences. As before, 
there are $\mathbf{p}$-steps where
a DP is applied at the
position of an annotated symbol. These steps may
introduce new annotations. Moreover, 
between two $\mathbf{p}$-steps there can be
several $\mathbf{r}$-steps.

A step of the form $(\mathbf{pr})$ at position $\pi$ in \Cref{def:ADP-Rewriting} 
represents  
a $\mathbf{p}$- or an  
$\mathbf{r}$-step\linebreak (or both), where an  
$\mathbf{r}$-step is only possible 
if one later rewrites an annotated symbol at a position above $\pi$.
All annotations are kept during this step except for annotations of subterms
that correspond to variables of the applied rule. Here, the used VRF $\varphi$ determines which of
these annotations are kept and which are removed.
As an example,
with the canonical ADP $\ta(x) \to \tb(x)$  from
$\ADPairMain{\R_3}$ we can rewrite
$\tA(\tF) \tored{}{}{\ADPairMain{\R_3}}  \tb(\tF)$ as in \Cref{example:rel-var-repos-function}.
Here, we have $\pi =
\varepsilon$, $\flat(s|_\varepsilon) = \ta(\tf) = \ell \sigma$, $r =
\tb(x)$, and the VRF $\varphi$ with 
$\varphi(1) = 1$ such that the annotation of $\tF$ in $\tA$'s argument is
kept in the  argument of $\tb$.

A step of the form $(\mathbf{r})$
rewrites at the position of a non-annotated defined symbol,
and represents just an $\mathbf{r}$-step.  
Hence, we remove all annotations
from the right-hand side $r$ of the ADP.
However, we may have to keep the annotations inside the substitution,
hence we move them according to the VRF.
For example, we obtain the rewrite step
$\ts(\tD(\underline{\tminus(\ts(\O),\ts(\O))},\ts(\O))) \tored{}{}{\ADPairMain{\R_\tdivl}}\ts(\tD(\tminus(\O,\O),\ts(\O)))$
using the ADP $\tminus(\ts(x),\ts(y)) 
\to \tM(x,y) \;$ \eqref{R-div-adp-1}
and any VRF.

A \emph{(relative) ADP problem} has the form
$(\C{P},\C{P}^{=})$, where $\C{P}$ and $\C{P}^{=}$ are finite sets of ADPs and $\C{P}^{=}$
is non-duplicating. 
$\C{P}$ is the set of all main ADPs and $\C{P}^{=}$ is the set of all base ADPs.
Now we can define chains in the relative setting.

\begin{definition}[Chains and Terminating ADP Problems]\label{def:relative-rewrite-chain}
    Let $(\C{P},\C{P}^{=})$ be an ADP problem.
    A sequence of terms $t_0, t_1, \ldots$ is a
    $(\C{P},\C{P}^{=})$-\defemph{chain} if we have $t_i \tored{}{}{\C{P} \cup \C{P}^{=}}
    t_{i+1}$ for all $i \in \IN$.
    The chain is called \defemph{infinite} if infinitely
    many of these rewrite steps use $\tored{}{}{\C{P}}$
    with Case $(\mathbf{pr})$. 
    We say that an ADP problem $(\C{P},\C{P}^{=})$ is \defemph{strongly normalizing (SN)} if
    there is no infinite $(\C{P},\C{P}^{=})$-chain.
\end{definition}

Note the two different forms of relativity in \Cref{def:relative-rewrite-chain}:
In a finite chain,
we may not only use infinitely many steps with $\C{P}^{=}$ but also infinitely many steps
with $\C{P}$ where Case $(\mathbf{r})$ applies.  
Thus, an ADP problem $(\C{P},\C{P}^{=})$
without annotated symbols or without any main ADPs (i.e., where $\C{P} = \emptyset$) is obviously SN.
Finally, we obtain our desired chain criterion.

\begin{restatable}[Chain Criterion for Relative Rewriting]{theorem}{RelChainCriterion}\label{theorem:relative-chain-crit}
    Let $\R$ and $\R^{=}$ be TRSs such that $\R^{=}$ is non-duplicating.
    Then $\R / \R^{=}$ is SN iff the ADP problem $(\ADPairMain{\R}, \ADPairBase{\R^{=}})$ is SN.
\end{restatable}

\begin{example}\label{ex:ADPs-for-redex-creation-1}
    The infinite rewrite sequence of \Cref{example:redex-creating} can be simulated by
    the following infinite chain  using $\ADPairMain{\R_2} = \{ \ta
    \to \tb\}$
    and $\ADPairBase{\R_2^{=}} = \{\tf \to
    \tc(\tF,\tA)\}$.
    \[\underline{\tF}
    \tored{}{}{\ADPairBase{\R_2^{=}}} \tc(\tF,\underline{\tA}) \tored{}{}{\ADPairMain{\R_2}} \tc(\underline{\tF},\tb)
    \tored{}{}{\ADPairBase{\R_2^{=}}} \tc(\tc(\tF,\underline{\tA}),\tb) \tored{}{}{\ADPairMain{\R_2}}
    \ldots\]

    The steps with $\tored{}{}{\ADPairBase{\R_2^{=}}}$ use Case
    ($\mathbf{pr}$) at the position of the annotated symbol $\tF$
    and the steps 
    with $\tored{}{}{\ADPairMain{\R_2}}$ use ($\mathbf{pr}$) as well.
    For this infinite chain, we indeed need 
    two annotated symbols in the right-hand side of the base ADP: If $\tA$ were not annotated (i.e., if we had the ADP
    $\tf \to
    \tc(\tF,\ta)$), then the step with 
    $\tored{}{}{\ADPairMain{\R_2}}$ would just use Case ($\mathbf{r}$) and the chain would
    not be considered ``infinite''. If $\tF$ were not annotated
    (i.e., if we had the ADP
    $\tf \to
    \tc(\tf,\tA)$), then we would have the step
     $\tf
    \tored{}{}{\ADPairBase{\R_2^{=}}} \tc(\tf,\ta)$ which uses
    Case ($\mathbf{r}$)
    and removes all
    annotations from the right-hand side. Hence, again the chain would not be considered ``infinite''.
\end{example}

\begin{example}\label{ex:ADPs-for-redex-creation-2}
  The infinite rewrite sequence of
  \Cref{example:redex-creatingAbove} is simulated by the following chain with
   $\ADPairMain{\R_3} = \{ \ta(x) \to \tb(x)\}$
    and $\ADPairBase{\R_3^{=}} = \{\tf \to \tA(\tF)\}$.
    \[\underline{\tF}
    \tored{}{}{\ADPairBase{\R_3^{=}}} \underline{\tA}(\tF) \tored{}{}{\ADPairMain{\R_3}} \tb(\underline{\tF})
    \tored{}{}{\ADPairBase{\R_3^{=}}} \tb(\underline{\tA}(\tF)) \tored{}{}{\ADPairMain{\R_3}}\tb(\tb(\underline{\tF}))  \tored{}{}{\ADPairBase{\R_3^{=}}}
    \ldots\]
    Here, it is important to use the VRF $\varphi(1) = 1$ for $\ta(x) \to \tb(x)$
    which keeps the annotation of $\tA$'s argument
    $\tF$ during the rewrite steps with $\ADPairMain{\R_3}$, i.e., these steps must yield
    $\tb(\tF)$ instead of $\tb(\tf)$    
    to generate further  subterms  $\tA(\ldots)$ afterwards.
\end{example}
