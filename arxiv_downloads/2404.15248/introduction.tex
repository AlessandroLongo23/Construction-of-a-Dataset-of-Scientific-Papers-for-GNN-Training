\section{Introduction}\label{Introduction}

Termination is an important topic in program verification.
There is a wealth of work on
automatic termination analysis of term rewrite systems (TRSs) which can also be used to
analyze termination of programs in many other languages.
Essentially all current termination tools for TRSs (e.g., \aprove~\cite{JAR-AProVE2017}, \natt~\cite{natt_sys_2014}, \muterm~\cite{gutierrez_mu-term_2020}, \ttttwo~\cite{ttt2_sys}, etc.)
use \emph{dependency pairs (DPs)}
\cite{arts2000termination,gieslLPAR04dpframework,giesl2006mechanizing,hirokawa2005automating,DBLP:journals/iandc/HirokawaM07}.


A combination of two TRSs $\R$ and $\R^{=}$
is considered to be ``\emph{relatively terminating}'' if there is no 
rewrite sequence that uses infinitely many steps
with rules from $\R$ (whereas rules from $\R^{=}$ may be used infinitely often).
Relative termination of TRSs has been
studied since decades \cite{DBLP:phd/dnb/Geser90}, and approaches based on
relative rewriting are used for many different applications, e.g.,
\cite{koprowskiZantema2005LivenessRelRew,fuhs2019DerivToRuntime,DBLP:journals/lmcs/NageleFZ17,DBLP:journals/corr/ZanklK14,DBLP:journals/jar/HirokawaM11,DBLP:conf/lpar/KleinH12,DBLP:conf/lopstr/IborraNV09,DBLP:journals/aaecc/NishidaV10,DBLP:conf/flops/Vidal08}.

However, while techniques and tools for analyzing ordinary termination of  TRSs
are very powerful due to the use of DPs, 
most approaches for automated analysis of relative termination are quite restricted in
power. Therefore, 
one of the largest open problems regarding DPs is Problem \#106 of the RTA List of Open Problems \cite{RTALoop}: 
\emph{Can we use the dependency pair method to prove relative termination?}
A first major step towards an answer to this question was presented in
\cite{iborra2017relative} by giving  criteria for $\R$ and $\R^{=}$ that allow the use of
ordinary DPs for relative termination. 

Recently, we adapted DPs  in order to analyze probabilistic innermost term rewriting, by using so-called
\emph{annotated dependency pairs (ADPs)} \cite{FLOPS2024}
or \emph{dependency tuples (DTs)} \cite{kassinggiesl2023iAST} (which were originally proposed
for innermost complexity analysis of  TRSs \cite{noschinski2013analyzing}).\footnote{As shown in
\cite{FLOPS2024}, using ADPs instead of DTs leads to a more elegant,
more powerful, and less complicated framework, and to completeness of
the underlying \emph{chain criterion}.}
In these adaptions, one \pagebreak[2] considers all \emph{defined} function symbols in the\linebreak[2]
right-hand side
of a rule at once, whereas ordinary DPs 
consider them separately.

In this paper, we show that considering the defined symbols on right-hand sides
separately (as for DPs) does not suffice for relative termination. On the other hand,
we do not need to consider all of them at once either.
Instead, we introduce a new definition of ADPs that is suitable for relative
termination and develop a corresponding ADP framework for automated relative
termination proofs of TRSs.
Moreover, while ADPs and DTs were only applicable for \emph{innermost}  rewriting in
\cite{FLOPS2024,kassinggiesl2023iAST,noschinski2013analyzing}, we now adapt ADPs to \emph{full}
(relative) rewriting, i.e., we do not impose any specific evaluation strategy.
So while \cite{iborra2017relative} presented conditions under which the \emph{ordinary
classical} DP framework can be used to 
prove relative termination, in this paper we develop the first \emph{specific} DP
framework for relative termination.

\medskip

\noindent
\textbf{Structure:} We start with preliminaries on relative rewriting in
\Cref{Relative Term Rewriting}.
In \Cref{DP Framework} we recapitulate the core processors of the DP framework.
Moreover, we state the main results of \cite{iborra2017relative} on using ordinary
DPs for relative termination.
Afterwards, we introduce our novel notion of \emph{annotated dependency pairs} for
relative termination in \Cref{Relative DP Framework}
and present a corresponding new ADP
framework in \Cref{Relative ADP Processors}.
We implemented our framework in the tool \aprove{} and
in  \Cref{Evaluation and Conclusion}, we evaluate our implementation in comparison to
other state-of-the-art  tools.
 All proofs can be found in \Cref{Appendix}.
