\section{Main Evaluation}\label{sec:experiment}
\newcommand{\markclosed}{{\color[HTML]{FE0000} Private}}

\begin{table*}[h]
\centering
\begin{tabular}{@{}lccccc@{}}
\toprule
\multirow{2}{*}{Model}  & \multicolumn{1}{c}{\multirow{2}{*}{Size}} & \multirow{2}{*}{\begin{tabular}[c]{@{}c@{}}Instruction\\ Dataset\end{tabular}} & \multirow{2}{*}{\begin{tabular}[c]{@{}c@{}}Dataset\\ Size\end{tabular}} & \multicolumn{2}{c}{Benchmark}                                    \\ \cmidrule(l){5-6} 
                        & \multicolumn{1}{c}{}                      & \multicolumn{1}{c}{}  & \multicolumn{1}{c}{}                                                                               & \multicolumn{1}{c}{HumanEval (+)} & \multicolumn{1}{c}{MBPP (+)} \\ \midrule
\gptthreefive (May 2023) & -         & {\markclosed} & -         & 73.2 (66.5)    & -            \\ \midrule
\stablecoder             & 3B        & -             & -         & 28.7 (25.6)    & 53.6 (44.1)  \\
\dscoderbase             & 1.3B      & -             & -         & 28.7 (25.6)    & 55.6 (46.9)  \\
Phi-2                    & 2.7B      & -             & -         & 48.8 (45.1)    & 62.7 (52.9)  \\
\dscoderinst             & 1.3B      & {\markclosed} & 2B        & 65.2 (59.8)    & 63.9 (53.1)  \\ \midrule
\baselineds                & 1.3B             &  Evol-Instruct             &        0.3B             & 61.6 (57.3)    & 59.6 (49.1)  \\
\ewads                     & 1.3B             & Evol-Instruct            &        0.3B              &   \textbf{67.1} (63.4)  &   58.9 (48.4)  \\ \midrule
\oursmoe                   & 8$\times$1.3B    & Evol-Instruct           &        0.3B              & 65.2 (62.2)    & 60.4 (50.1)  \\
\oursmerge                 & 1.3B             & Evol-Instruct           &        0.3B               & \textbf{67.1} (\textbf{64.6})    & \textbf{60.4} (\textbf{50.1})  \\ \bottomrule
\end{tabular}
\caption{\label{tab:python-text2code}
\Passat{1} results of different \llm{s} on \humaneval{}~(+) and \mbpp{}~(+) computed with greedy decoding, following the setting of prior works~\cite{wei2023magicoder, evalplus}. We report the results consistently from the \evalplus~\cite{evalplus} Leaderboard. Note that numbers in bold refer to the highest scores among all 1.3B models fine-tuned on public datasets, which is the same for all the other tables.
}
\end{table*}


\subsection{Experimental Setup}
\textbf{Training.} 
We use \dscoderbase 1.3B~\cite{guo2024deepseekcoder} as the main base code \llm. 
\evolcode, an open-source \evolinstruct~\cite{luo2023wizardcoder} dataset containing 110K samples, is used as our instruction dataset.
\textbf{\oursmoe}, our \moe model upcycled from the base model, is implemented following Llama-MoE~\cite{llama-moe-2023}. 
It is constructed with 8 experts in one expert layer and the top 6 experts\footnote{
6 is the best-performing number of activated experts per our \humanevalp{} experiments using top $\{2,4,6\}$ experts.} are activated for each token, including one shared expert. 
As such, we denote the model size of \textbf{\oursmoe} as 8$\times$1.3B.
Other hyperparameter settings are detailed in Appendix \ref{sec:hyperparameter}. 
We finally obtain \textbf{\oursmerge} by using the learned mixing coefficients to merge \moe layers inside \oursmoe as normal FFN layers. 
Note that \textbf{\oursmerge} is the final instruction-tuned \llm we produce, while \textbf{\oursmoe} is only an intermediate product of \ours framework.


\textbf{Baselines.} 
To study the effectiveness of \ours, we build a baseline model, namely \textbf{\baselineds}, by directly performing SFT for \dscoderbase 1.3B on \evolcode. 
To compare \ours with \ewa~\cite{huang2023experts}, we also implement a baseline \textbf{\ewads} and instruction-tune it using the same hyperparameter setting as \baselineds, which is described in Appendix \ref{sec:hyperparameter}. More implementation details of \ewads can be seen in Appendix \ref{sec:ewa_details}. 
Furthermore, we incorporate multiple small open-source models (<3B) as our baselines, including \dscoderbase 1.3B, \dscoderinst 1.3B~\cite{guo2024deepseekcoder}, Phi-2 2.7B, and \stablecoder 3B~\cite{stable-code-3b}.


\begin{table*}[h]
\centering
\begin{tabular}{@{}lcrrrrrrr@{}}
\toprule
\multirow{2}{*}{Model} & \multirow{2}{*}{Size} & \multicolumn{6}{c}{Programming Language}                                                      & \multirow{2}{*}{\textbf{Average}} \\ \cmidrule(lr){3-8}
                       &                       & C++           & PHP           & Java          & \multicolumn{1}{c}{JS}    & Swift         & Rust          &                          \\ \midrule
\dscoderbase    & 1.3B                  & 28.1          & 22.9         & 27.2          & 28.7           & 10.9          & 18.0          & 22.6                     \\ \midrule
\baselineds                 & 1.3B                  & 40.4          & 38.5          & \textbf{40.2} & 46.2          & 16.4          & 27.7          & 34.9                     \\
\ewads                    & 1.3B                  &     39.4      &     38.4      &      37.3     &      45.2     &     20.9      &   28.6   &    35.0                 \\ \midrule
\oursmoe                  & 8$\times$1.3B                & 42.2          & 42.2 & 35.4          & 49.8          & 24.7          & 30.6          & 37.5                     \\
\oursmerge                  & 1.3B                  & \textbf{42.7} & \textbf{41.5} & 36.0          & \textbf{49.7} & \textbf{25.3} & \textbf{32.1}          & \textbf{37.9}            \\ \bottomrule
\end{tabular}
\caption{\label{tab:multilang}
\Passat{1} results on \multiple~\cite{cassano2022multiple} following the same hyperparameter settings as prior works~\cite{wei2023magicoder, luo2023wizardcoder}: $\temperature=0.2$, $\topp=0.95$, $\maxLen=512$, and $\nsamples=50$. All models are evaluated using   \bigcodeharness{}~\cite{bigcode-evaluation-harness}.
}
\end{table*}

\begin{table*}[h]
\centering
\begin{tabular}{@{}lcrrrrrrrr@{}}
\toprule
\multirow{2}{*}{Model} & \multirow{2}{*}{Size} & \multicolumn{7}{c}{Data Science Library}                                                                      & \multirow{2}{*}{\textbf{Overall}} \\ \cmidrule(lr){3-9}
                       &                       &  \multicolumn{1}{c}{np}         &  \multicolumn{1}{c}{pd}        &  \multicolumn{1}{c}{plt}    &  \multicolumn{1}{c}{py}       &  \multicolumn{1}{c}{scp}         &  \multicolumn{1}{c}{tf}    &  \multicolumn{1}{c}{sk}       &                          \\ \midrule
\dscoderbase    & 1.3B                  &       25.1        &       5.8        &        34.5       &       12.7        &      9.8         &       11.1        &       12.7        &        16.4                  \\ \midrule
\baselineds                 & 1.3B                  & 30.9          & 17.0          & 40.5 & 32.7          & 18.3          & 21.1 & 24.4          & 25.9                     \\
\ewads                    & 1.3B                  &   32.9   &    19.4         &   \textbf{41.8}  &       25.7       &      17.7      &       \textbf{22.2}        &       33.0        &    27.8                     \\ \midrule
\oursmoe                  & 8$\times$1.3B                & 33.2          & 21.3          & 38.4          & 41.8          & 21.8          & 23.5    & 37.5          & 30.0               \\
\oursmerge                  & 1.3B                  & \textbf{32.9} & \textbf{20.2} & 38.9          & \textbf{41.4} & \textbf{21.1} & 16.9          & \textbf{37.5} & \textbf{29.3}            \\ \bottomrule
\end{tabular}
\caption{\label{tab:ds1000}
\Passat{1} results on \dsonek{} (completion format) with $\temperature=0.2$, $\topp=0.5$, $\maxLen=1024$, and $\nsamples=40$, following the same hyperparameter setting used in prior works~\cite{wei2023magicoder}.
}
\end{table*}

\subsection{Python Text-to-Code Generation}

\humaneval~\cite{chen2021evaluating} and \mbpp~\cite{austin2021program} benchmarks are the two most widely-used collections of Python code generation tasks. 
We further employ \humanevalp and \mbppp, which use more tests automatically generated by \evalplus~\cite{evalplus} for more rigorous evaluation. 
We leave the details in Appendix \ref{sec:benchmarks}.

Table~\ref{tab:python-text2code} shows the \passat{1} results of different \llm{s}.
\ours achieves 67.1 \passat{1} on \humaneval and 64.6 \passat{1} on \humanevalp, which makes it the new state-of-the-art small code \llm{} (<3B). 
We can also observe that \oursmerge has a clear improvement over the \baselineds on both benchmarks, with 13\% and 2\% improvement on \humanevalp and \mbppp respectively, while \ewads even performs worse than \baselineds on \mbpp{(+)}.
\oursmerge also outperforms \ewads on both benchmarks. 
Surprisingly, \oursmerge even surpasses \oursmoe on \humaneval and \humanevalp, despite only using around $\sfrac{1}{8}\times$ parameters and around $\sfrac{1}{6}\times$ computations, which showcases the effectiveness of our simple learnable merging technique. Appendix~\ref{sec:statistic} further demonstrates the statistical significance of the improvements brought by \ours.

\subsection{Multilingual Code Generation}\label{sec:multiple}

We use \multiple~\cite{cassano2022multiple}, a multi-programming benchmark that supports 18 programming languages in addition to Python, to evaluate the multilingual ability and generalizability of \ours{}. 
Among these, we choose 6 representative programming for their distinct language features: Java, JavaScript, C++, PHP, Swift, and Rust, following~\citet{wei2023magicoder}.
Table \ref{tab:multilang} shows, among all 1.3B models, \oursmerge achieves the best average multilingual performance and performs the best on 5 (out of 6) individual programming languages, 
overall largely improving \baselineds{} which uses standard SFT.
Notably, the overall performance of \ewads is on par with \baselineds, indicating that \ewads{} may not improve SFT on multilingual coding.
Appendix~\ref{sec:expert_analysis} further studies whether each expert in \oursmoe specializes differently in these programming languages.

\subsection{Code Generation for Data Science}

The \dsonek dataset~\cite{lai2022ds1000} is a collection of 1000 realistic data science coding problems ranging from 7 popular data science libraries in Python, including Matplotlib (plt), NumPy (np), Pandas (pd), SciPy (scp), Scikit-Learn (sk), PyTorch (py), and TensorFlow (tf). 
We evaluate \ours on \dsonek{} to understand its effectiveness for practical data science engineering. 
We follow the evaluation setting of prior works~\cite{guo2024deepseekcoder, wei2023magicoder}. 
In Table \ref{tab:ds1000}, \oursmerge achieves the best overall performance among all the evaluated 1.3B models. 
Specifically, \oursmerge consistently surpasses \baselineds among all the seven studied libraries and also outperforms \ewads in general.


