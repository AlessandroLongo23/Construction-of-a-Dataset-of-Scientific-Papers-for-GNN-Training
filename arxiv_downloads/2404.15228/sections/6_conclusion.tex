\section{Conclusion}
In this work, we investigated the ability of LLMs to solve inverse-graphics challenges.
Introducing the Inverse-Graphics Large-Language-Model (IG-LLM) framework, we demonstrated that the broad generalization and reasoning capabilities of LLMs can be harnessed to facilitate inverse-graphics tasks.
Through extensive evaluation, we assessed the model's capacity to generalize out-of-domain, revealing its ability to abstract scene elements compositionally.
We additionally explored the integration of a numeric head to adapt LLMs for continuous metric-value estimation, providing enhanced generalization and smoother training dynamics.
Our quantitative analyses demonstrate its ability to generalize compositionally (\cref{ssec:clevr}), in parameter space (\cref{ssec:parameter_space_generalization}), and across visual domains (\cref{ssec:6dof}).
Our investigation demonstrates the ability of IG-LLM to leverage the general knowledge of LLMs in solving inverse-graphics problems, opening a new avenue for research.