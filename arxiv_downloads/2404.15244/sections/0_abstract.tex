\begin{abstract}
    Vision transformer based models bring significant improvements for image segmentation tasks. Although these architectures offer powerful capabilities irrespective of specific segmentation tasks, their use of computational resources can be taxing on deployed devices.  One way to overcome this challenge is by adapting the computation level to the specific needs of the input image rather than the current one-size-fits-all approach. To this end, we introduce \ours or \textbf{E}ffi\textbf{C}ient Transf\textbf{O}rmer Encoders for \textbf{M}ask\textbf{2F}ormer-style models. Noting that the encoder module of M2F-style models incur high resource-intensive computations, \ours provides a strategy to self-select the number of hidden layers in the encoder, conditioned on the input image. To enable this self-selection ability for providing a balance between performance and computational efficiency, we present a three step recipe. The \textit{first} step is to train the parent architecture to enable early exiting from the encoder. The \textit{second} step is to create an derived dataset of the ideal number of encoder layers required for each training example. The \textit{third} step is to use the aforementioned derived dataset to train a gating network that predicts the number of encoder layers to be used, conditioned on input image. Additionally, to change the computational-accuracy trade-off, only steps two and three need to be repeated which significantly reduces retraining time. Experiments on the public datasets show that the proposed approach reduces expected encoder computational cost while maintaining performance, adapts to various user compute resources, is flexible in architecture configurations, and can be extended beyond the segmentation task to object detection. % Code can be found here: \url{https://github.com/GYeow/ECO-M2F}
\end{abstract}
