%%
%% The abstract is a short summary of the work to be presented in the
%% article.
\begin{abstract}
Recent advancements in controllable human image generation have led to zero-shot generation using structural signals (e.g., pose, depth) or facial appearance. Yet, generating human images conditioned on images of individual body parts remains challenging. Addressing this, we introduce Parts2Whole, a novel framework designed for generating customized portraits from multiple reference images, including pose images and various aspects of human appearance. To achieve this, we first develop a semantic-aware appearance encoder to retain details of different human parts, which processes each image based on its textual label to a series of multi-scale feature maps rather than one image token. Second, our framework supports multi-image conditioned generation through a shared attention mechanism that operates across reference and target features during the diffusion process. We enhance the vanilla attention mechanism by incorporating mask information from the subject in the reference images, allowing for more precise selection. Extensive experiments demonstrate the superiority of our approach over existing alternatives, offering advanced capabilities for multi-part controllable human image customization.
\end{abstract}

%%
%% The code below is generated by the tool at http://dl.acm.org/ccs.cfm.
%% Please copy and paste the code instead of the example below.
%%


\begin{CCSXML}
<ccs2012>
   <concept>
       <concept_id>10010147.10010178</concept_id>
       <concept_desc>Computing methodologies~Artificial intelligence</concept_desc>
       <concept_significance>500</concept_significance>
       </concept>
   <concept>
       <concept_id>10010147.10010178.10010224</concept_id>
       <concept_desc>Computing methodologies~Computer vision</concept_desc>
       <concept_significance>500</concept_significance>
       </concept>
   <concept>
       <concept_id>10010147.10010257</concept_id>
       <concept_desc>Computing methodologies~Machine learning</concept_desc>
       <concept_significance>300</concept_significance>
       </concept>
 </ccs2012>
\end{CCSXML}

\ccsdesc[500]{Computing methodologies~Artificial intelligence}
\ccsdesc[500]{Computing methodologies~Computer vision}
\ccsdesc[300]{Computing methodologies~Machine learning}




%%
%% Keywords. The author(s) should pick words that accurately describe
%% the work being presented. Separate the keywords with commas.
\keywords{Controllable Human Image Generation, Latent Diffusion Models, Generative Architectures}