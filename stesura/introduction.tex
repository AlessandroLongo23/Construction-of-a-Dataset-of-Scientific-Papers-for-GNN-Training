\chapter*{Introduzione}
Lo scopo di questa tesi è la presentazione di un algoritmo il cui scopo è la generazione di un dataset per l'addestramento di una GNN. 
Il dataset generato contiene coppie composte ognuna da un file LaTeX (.tex) e dal file pdf (.pdf) generato a partire da esso: 
oltre a questo vengono generati dei dati per il matching tra i due file, che forniscono le informazioni per associare
ogni elemento presenti nel file di ouput (pdf) alla corrispondente porzione del file di input (LaTeX).

In questo modo, ognuno di essi può essere trattato come un nodo della GNN e le relazioni tra essi, come per esempio l'appartenenza
alla stessa pagina o allo stesso blocco di testo, possono essere rappresentate con degli archi.

\section*{Struttura della tesi}
    \subsection*{Capitolo 1}
    Nel primo capitolo verrà introdotto il problema del DLA e alcune delle soluzioni che sono state impiegate nel corso degli anni per risolverlo.

    \subsection*{Capitolo 2}
    Il secondo capitolo presenta una descrizione ad alto livello dell'algoritmo proposto e del suo processo di stesura, nonché la discussione di
    alcune scelte implementative e di possibili alternative.

    \subsection*{Capitolo 3}
    L'ultimo capitolo analizza i risultati ottenuti, assesta la qualità del matching nelle coppie prodotte e suggerisce degli spunti per possibili miglioramenti futuri. 