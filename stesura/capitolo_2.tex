\chapter{Il codice e l'algoritmo di matching}

\section{Linguaggio e risorse utilizzate}
	L'algoritmo che verrà ora presentato è stato scritto in Python 3.12 con l'aiuto di alcune librerie e moduli esterni.

\section{Classi}
	Il codice presenta quattro classi distinte:
	\begin{itemize}
		\item La classe WebScraper, per l'ottenimento dei file dal database online Arxiv.
		\item La classe latexData, per il parsing e l'analisi del file LaTeX.
		\item La classe PDFData, per il parsing e l'analisi del file pdf.
		\item La classe MatchingTool, che contiene il vero e proprio algoritmo di matching.
	\end{itemize}

	\subsection{WebScraper}
		La classe \textit{WebScraper}, tramite la libreria \textit{request}, accede all'archivio online di documenti scientifici Arxiv e tenta
		di effettuare il download delle coppie LaTeX-pdf per ogni articolo il cui codice identificativo rientra in un intervallo
		predeterminato dall'utente.
		Poiché il file sorgente LaTeX non è disponibile per tutti gli articoli nel database, vengono mantenute solo le 
		coppie che presentano entrambi i file. Per questi articoli, la cartella compressa contenente i file sorgente viene 
		estratta, il file sorgente principale viene individuato e, insieme al pdf, sono rinominati e organizzati in cartelle
		in modo da facilitare l'indicizzazione e il reperimento successivi.

	\subsection{LatexData}
		La classe \textit{LatexData} si occupa di processare il file sorgente e di prepararne il contenuto
		per il matching. I passi includono:
		\begin{itemize}
			\item Sostituzione dei comandi "include" e "input" con il contenuto dei file chiamati
			\item Parsing e preprocessamento del documento
			\item Creazione di un albero gerarchico del contenuto
			\item Enumerazione e delle foglie di tale albero, corrispondenti ai singoli blocchi di contenuto (paragrafi, formule, tabelle, ecc.)
		\end{itemize}

	\subsection{PDFData}
		La classe \textit{PDFData} si occupa di processare, tramite l'utilizzo della libreria pdfminer, il file
		di output. Dopo aver estratto tutti gli elementi (linee di testo, immagini, ecc.), questi vengono salvati in una struttura dati
		che contiene, tra gli altri, il contenuto testuale, se sono di tipo \textit{pdfminer.LTTextBox}, le coordinate della bounding box e la pagina 
		di appartenenza.

	\subsection{MatchingTool}
		La classe \textit{MatchingTool} è responsabile dell'accoppiamento tra i due file e contiene l'algoritmo vero e proprio.
		Questo si basa principalmente sull'algoritmo LCS, per la comparazione delle caselle di testo del pdf e le porzioni del file LaTeX.

\section{L'algoritmo di matching}
	\subsection{}





